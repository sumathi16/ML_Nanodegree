
% Default to the notebook output style

    


% Inherit from the specified cell style.




    
\documentclass[11pt]{article}

    
    
    \usepackage[T1]{fontenc}
    % Nicer default font (+ math font) than Computer Modern for most use cases
    \usepackage{mathpazo}

    % Basic figure setup, for now with no caption control since it's done
    % automatically by Pandoc (which extracts ![](path) syntax from Markdown).
    \usepackage{graphicx}
    % We will generate all images so they have a width \maxwidth. This means
    % that they will get their normal width if they fit onto the page, but
    % are scaled down if they would overflow the margins.
    \makeatletter
    \def\maxwidth{\ifdim\Gin@nat@width>\linewidth\linewidth
    \else\Gin@nat@width\fi}
    \makeatother
    \let\Oldincludegraphics\includegraphics
    % Set max figure width to be 80% of text width, for now hardcoded.
    \renewcommand{\includegraphics}[1]{\Oldincludegraphics[width=.8\maxwidth]{#1}}
    % Ensure that by default, figures have no caption (until we provide a
    % proper Figure object with a Caption API and a way to capture that
    % in the conversion process - todo).
    \usepackage{caption}
    \DeclareCaptionLabelFormat{nolabel}{}
    \captionsetup{labelformat=nolabel}

    \usepackage{adjustbox} % Used to constrain images to a maximum size 
    \usepackage{xcolor} % Allow colors to be defined
    \usepackage{enumerate} % Needed for markdown enumerations to work
    \usepackage{geometry} % Used to adjust the document margins
    \usepackage{amsmath} % Equations
    \usepackage{amssymb} % Equations
    \usepackage{textcomp} % defines textquotesingle
    % Hack from http://tex.stackexchange.com/a/47451/13684:
    \AtBeginDocument{%
        \def\PYZsq{\textquotesingle}% Upright quotes in Pygmentized code
    }
    \usepackage{upquote} % Upright quotes for verbatim code
    \usepackage{eurosym} % defines \euro
    \usepackage[mathletters]{ucs} % Extended unicode (utf-8) support
    \usepackage[utf8x]{inputenc} % Allow utf-8 characters in the tex document
    \usepackage{fancyvrb} % verbatim replacement that allows latex
    \usepackage{grffile} % extends the file name processing of package graphics 
                         % to support a larger range 
    % The hyperref package gives us a pdf with properly built
    % internal navigation ('pdf bookmarks' for the table of contents,
    % internal cross-reference links, web links for URLs, etc.)
    \usepackage{hyperref}
    \usepackage{longtable} % longtable support required by pandoc >1.10
    \usepackage{booktabs}  % table support for pandoc > 1.12.2
    \usepackage[inline]{enumitem} % IRkernel/repr support (it uses the enumerate* environment)
    \usepackage[normalem]{ulem} % ulem is needed to support strikethroughs (\sout)
                                % normalem makes italics be italics, not underlines
    

    
    
    % Colors for the hyperref package
    \definecolor{urlcolor}{rgb}{0,.145,.698}
    \definecolor{linkcolor}{rgb}{.71,0.21,0.01}
    \definecolor{citecolor}{rgb}{.12,.54,.11}

    % ANSI colors
    \definecolor{ansi-black}{HTML}{3E424D}
    \definecolor{ansi-black-intense}{HTML}{282C36}
    \definecolor{ansi-red}{HTML}{E75C58}
    \definecolor{ansi-red-intense}{HTML}{B22B31}
    \definecolor{ansi-green}{HTML}{00A250}
    \definecolor{ansi-green-intense}{HTML}{007427}
    \definecolor{ansi-yellow}{HTML}{DDB62B}
    \definecolor{ansi-yellow-intense}{HTML}{B27D12}
    \definecolor{ansi-blue}{HTML}{208FFB}
    \definecolor{ansi-blue-intense}{HTML}{0065CA}
    \definecolor{ansi-magenta}{HTML}{D160C4}
    \definecolor{ansi-magenta-intense}{HTML}{A03196}
    \definecolor{ansi-cyan}{HTML}{60C6C8}
    \definecolor{ansi-cyan-intense}{HTML}{258F8F}
    \definecolor{ansi-white}{HTML}{C5C1B4}
    \definecolor{ansi-white-intense}{HTML}{A1A6B2}

    % commands and environments needed by pandoc snippets
    % extracted from the output of `pandoc -s`
    \providecommand{\tightlist}{%
      \setlength{\itemsep}{0pt}\setlength{\parskip}{0pt}}
    \DefineVerbatimEnvironment{Highlighting}{Verbatim}{commandchars=\\\{\}}
    % Add ',fontsize=\small' for more characters per line
    \newenvironment{Shaded}{}{}
    \newcommand{\KeywordTok}[1]{\textcolor[rgb]{0.00,0.44,0.13}{\textbf{{#1}}}}
    \newcommand{\DataTypeTok}[1]{\textcolor[rgb]{0.56,0.13,0.00}{{#1}}}
    \newcommand{\DecValTok}[1]{\textcolor[rgb]{0.25,0.63,0.44}{{#1}}}
    \newcommand{\BaseNTok}[1]{\textcolor[rgb]{0.25,0.63,0.44}{{#1}}}
    \newcommand{\FloatTok}[1]{\textcolor[rgb]{0.25,0.63,0.44}{{#1}}}
    \newcommand{\CharTok}[1]{\textcolor[rgb]{0.25,0.44,0.63}{{#1}}}
    \newcommand{\StringTok}[1]{\textcolor[rgb]{0.25,0.44,0.63}{{#1}}}
    \newcommand{\CommentTok}[1]{\textcolor[rgb]{0.38,0.63,0.69}{\textit{{#1}}}}
    \newcommand{\OtherTok}[1]{\textcolor[rgb]{0.00,0.44,0.13}{{#1}}}
    \newcommand{\AlertTok}[1]{\textcolor[rgb]{1.00,0.00,0.00}{\textbf{{#1}}}}
    \newcommand{\FunctionTok}[1]{\textcolor[rgb]{0.02,0.16,0.49}{{#1}}}
    \newcommand{\RegionMarkerTok}[1]{{#1}}
    \newcommand{\ErrorTok}[1]{\textcolor[rgb]{1.00,0.00,0.00}{\textbf{{#1}}}}
    \newcommand{\NormalTok}[1]{{#1}}
    
    % Additional commands for more recent versions of Pandoc
    \newcommand{\ConstantTok}[1]{\textcolor[rgb]{0.53,0.00,0.00}{{#1}}}
    \newcommand{\SpecialCharTok}[1]{\textcolor[rgb]{0.25,0.44,0.63}{{#1}}}
    \newcommand{\VerbatimStringTok}[1]{\textcolor[rgb]{0.25,0.44,0.63}{{#1}}}
    \newcommand{\SpecialStringTok}[1]{\textcolor[rgb]{0.73,0.40,0.53}{{#1}}}
    \newcommand{\ImportTok}[1]{{#1}}
    \newcommand{\DocumentationTok}[1]{\textcolor[rgb]{0.73,0.13,0.13}{\textit{{#1}}}}
    \newcommand{\AnnotationTok}[1]{\textcolor[rgb]{0.38,0.63,0.69}{\textbf{\textit{{#1}}}}}
    \newcommand{\CommentVarTok}[1]{\textcolor[rgb]{0.38,0.63,0.69}{\textbf{\textit{{#1}}}}}
    \newcommand{\VariableTok}[1]{\textcolor[rgb]{0.10,0.09,0.49}{{#1}}}
    \newcommand{\ControlFlowTok}[1]{\textcolor[rgb]{0.00,0.44,0.13}{\textbf{{#1}}}}
    \newcommand{\OperatorTok}[1]{\textcolor[rgb]{0.40,0.40,0.40}{{#1}}}
    \newcommand{\BuiltInTok}[1]{{#1}}
    \newcommand{\ExtensionTok}[1]{{#1}}
    \newcommand{\PreprocessorTok}[1]{\textcolor[rgb]{0.74,0.48,0.00}{{#1}}}
    \newcommand{\AttributeTok}[1]{\textcolor[rgb]{0.49,0.56,0.16}{{#1}}}
    \newcommand{\InformationTok}[1]{\textcolor[rgb]{0.38,0.63,0.69}{\textbf{\textit{{#1}}}}}
    \newcommand{\WarningTok}[1]{\textcolor[rgb]{0.38,0.63,0.69}{\textbf{\textit{{#1}}}}}
    
    
    % Define a nice break command that doesn't care if a line doesn't already
    % exist.
    \def\br{\hspace*{\fill} \\* }
    % Math Jax compatability definitions
    \def\gt{>}
    \def\lt{<}
    % Document parameters
    \title{customer\_segments}
    
    
    

    % Pygments definitions
    
\makeatletter
\def\PY@reset{\let\PY@it=\relax \let\PY@bf=\relax%
    \let\PY@ul=\relax \let\PY@tc=\relax%
    \let\PY@bc=\relax \let\PY@ff=\relax}
\def\PY@tok#1{\csname PY@tok@#1\endcsname}
\def\PY@toks#1+{\ifx\relax#1\empty\else%
    \PY@tok{#1}\expandafter\PY@toks\fi}
\def\PY@do#1{\PY@bc{\PY@tc{\PY@ul{%
    \PY@it{\PY@bf{\PY@ff{#1}}}}}}}
\def\PY#1#2{\PY@reset\PY@toks#1+\relax+\PY@do{#2}}

\expandafter\def\csname PY@tok@w\endcsname{\def\PY@tc##1{\textcolor[rgb]{0.73,0.73,0.73}{##1}}}
\expandafter\def\csname PY@tok@c\endcsname{\let\PY@it=\textit\def\PY@tc##1{\textcolor[rgb]{0.25,0.50,0.50}{##1}}}
\expandafter\def\csname PY@tok@cp\endcsname{\def\PY@tc##1{\textcolor[rgb]{0.74,0.48,0.00}{##1}}}
\expandafter\def\csname PY@tok@k\endcsname{\let\PY@bf=\textbf\def\PY@tc##1{\textcolor[rgb]{0.00,0.50,0.00}{##1}}}
\expandafter\def\csname PY@tok@kp\endcsname{\def\PY@tc##1{\textcolor[rgb]{0.00,0.50,0.00}{##1}}}
\expandafter\def\csname PY@tok@kt\endcsname{\def\PY@tc##1{\textcolor[rgb]{0.69,0.00,0.25}{##1}}}
\expandafter\def\csname PY@tok@o\endcsname{\def\PY@tc##1{\textcolor[rgb]{0.40,0.40,0.40}{##1}}}
\expandafter\def\csname PY@tok@ow\endcsname{\let\PY@bf=\textbf\def\PY@tc##1{\textcolor[rgb]{0.67,0.13,1.00}{##1}}}
\expandafter\def\csname PY@tok@nb\endcsname{\def\PY@tc##1{\textcolor[rgb]{0.00,0.50,0.00}{##1}}}
\expandafter\def\csname PY@tok@nf\endcsname{\def\PY@tc##1{\textcolor[rgb]{0.00,0.00,1.00}{##1}}}
\expandafter\def\csname PY@tok@nc\endcsname{\let\PY@bf=\textbf\def\PY@tc##1{\textcolor[rgb]{0.00,0.00,1.00}{##1}}}
\expandafter\def\csname PY@tok@nn\endcsname{\let\PY@bf=\textbf\def\PY@tc##1{\textcolor[rgb]{0.00,0.00,1.00}{##1}}}
\expandafter\def\csname PY@tok@ne\endcsname{\let\PY@bf=\textbf\def\PY@tc##1{\textcolor[rgb]{0.82,0.25,0.23}{##1}}}
\expandafter\def\csname PY@tok@nv\endcsname{\def\PY@tc##1{\textcolor[rgb]{0.10,0.09,0.49}{##1}}}
\expandafter\def\csname PY@tok@no\endcsname{\def\PY@tc##1{\textcolor[rgb]{0.53,0.00,0.00}{##1}}}
\expandafter\def\csname PY@tok@nl\endcsname{\def\PY@tc##1{\textcolor[rgb]{0.63,0.63,0.00}{##1}}}
\expandafter\def\csname PY@tok@ni\endcsname{\let\PY@bf=\textbf\def\PY@tc##1{\textcolor[rgb]{0.60,0.60,0.60}{##1}}}
\expandafter\def\csname PY@tok@na\endcsname{\def\PY@tc##1{\textcolor[rgb]{0.49,0.56,0.16}{##1}}}
\expandafter\def\csname PY@tok@nt\endcsname{\let\PY@bf=\textbf\def\PY@tc##1{\textcolor[rgb]{0.00,0.50,0.00}{##1}}}
\expandafter\def\csname PY@tok@nd\endcsname{\def\PY@tc##1{\textcolor[rgb]{0.67,0.13,1.00}{##1}}}
\expandafter\def\csname PY@tok@s\endcsname{\def\PY@tc##1{\textcolor[rgb]{0.73,0.13,0.13}{##1}}}
\expandafter\def\csname PY@tok@sd\endcsname{\let\PY@it=\textit\def\PY@tc##1{\textcolor[rgb]{0.73,0.13,0.13}{##1}}}
\expandafter\def\csname PY@tok@si\endcsname{\let\PY@bf=\textbf\def\PY@tc##1{\textcolor[rgb]{0.73,0.40,0.53}{##1}}}
\expandafter\def\csname PY@tok@se\endcsname{\let\PY@bf=\textbf\def\PY@tc##1{\textcolor[rgb]{0.73,0.40,0.13}{##1}}}
\expandafter\def\csname PY@tok@sr\endcsname{\def\PY@tc##1{\textcolor[rgb]{0.73,0.40,0.53}{##1}}}
\expandafter\def\csname PY@tok@ss\endcsname{\def\PY@tc##1{\textcolor[rgb]{0.10,0.09,0.49}{##1}}}
\expandafter\def\csname PY@tok@sx\endcsname{\def\PY@tc##1{\textcolor[rgb]{0.00,0.50,0.00}{##1}}}
\expandafter\def\csname PY@tok@m\endcsname{\def\PY@tc##1{\textcolor[rgb]{0.40,0.40,0.40}{##1}}}
\expandafter\def\csname PY@tok@gh\endcsname{\let\PY@bf=\textbf\def\PY@tc##1{\textcolor[rgb]{0.00,0.00,0.50}{##1}}}
\expandafter\def\csname PY@tok@gu\endcsname{\let\PY@bf=\textbf\def\PY@tc##1{\textcolor[rgb]{0.50,0.00,0.50}{##1}}}
\expandafter\def\csname PY@tok@gd\endcsname{\def\PY@tc##1{\textcolor[rgb]{0.63,0.00,0.00}{##1}}}
\expandafter\def\csname PY@tok@gi\endcsname{\def\PY@tc##1{\textcolor[rgb]{0.00,0.63,0.00}{##1}}}
\expandafter\def\csname PY@tok@gr\endcsname{\def\PY@tc##1{\textcolor[rgb]{1.00,0.00,0.00}{##1}}}
\expandafter\def\csname PY@tok@ge\endcsname{\let\PY@it=\textit}
\expandafter\def\csname PY@tok@gs\endcsname{\let\PY@bf=\textbf}
\expandafter\def\csname PY@tok@gp\endcsname{\let\PY@bf=\textbf\def\PY@tc##1{\textcolor[rgb]{0.00,0.00,0.50}{##1}}}
\expandafter\def\csname PY@tok@go\endcsname{\def\PY@tc##1{\textcolor[rgb]{0.53,0.53,0.53}{##1}}}
\expandafter\def\csname PY@tok@gt\endcsname{\def\PY@tc##1{\textcolor[rgb]{0.00,0.27,0.87}{##1}}}
\expandafter\def\csname PY@tok@err\endcsname{\def\PY@bc##1{\setlength{\fboxsep}{0pt}\fcolorbox[rgb]{1.00,0.00,0.00}{1,1,1}{\strut ##1}}}
\expandafter\def\csname PY@tok@kc\endcsname{\let\PY@bf=\textbf\def\PY@tc##1{\textcolor[rgb]{0.00,0.50,0.00}{##1}}}
\expandafter\def\csname PY@tok@kd\endcsname{\let\PY@bf=\textbf\def\PY@tc##1{\textcolor[rgb]{0.00,0.50,0.00}{##1}}}
\expandafter\def\csname PY@tok@kn\endcsname{\let\PY@bf=\textbf\def\PY@tc##1{\textcolor[rgb]{0.00,0.50,0.00}{##1}}}
\expandafter\def\csname PY@tok@kr\endcsname{\let\PY@bf=\textbf\def\PY@tc##1{\textcolor[rgb]{0.00,0.50,0.00}{##1}}}
\expandafter\def\csname PY@tok@bp\endcsname{\def\PY@tc##1{\textcolor[rgb]{0.00,0.50,0.00}{##1}}}
\expandafter\def\csname PY@tok@fm\endcsname{\def\PY@tc##1{\textcolor[rgb]{0.00,0.00,1.00}{##1}}}
\expandafter\def\csname PY@tok@vc\endcsname{\def\PY@tc##1{\textcolor[rgb]{0.10,0.09,0.49}{##1}}}
\expandafter\def\csname PY@tok@vg\endcsname{\def\PY@tc##1{\textcolor[rgb]{0.10,0.09,0.49}{##1}}}
\expandafter\def\csname PY@tok@vi\endcsname{\def\PY@tc##1{\textcolor[rgb]{0.10,0.09,0.49}{##1}}}
\expandafter\def\csname PY@tok@vm\endcsname{\def\PY@tc##1{\textcolor[rgb]{0.10,0.09,0.49}{##1}}}
\expandafter\def\csname PY@tok@sa\endcsname{\def\PY@tc##1{\textcolor[rgb]{0.73,0.13,0.13}{##1}}}
\expandafter\def\csname PY@tok@sb\endcsname{\def\PY@tc##1{\textcolor[rgb]{0.73,0.13,0.13}{##1}}}
\expandafter\def\csname PY@tok@sc\endcsname{\def\PY@tc##1{\textcolor[rgb]{0.73,0.13,0.13}{##1}}}
\expandafter\def\csname PY@tok@dl\endcsname{\def\PY@tc##1{\textcolor[rgb]{0.73,0.13,0.13}{##1}}}
\expandafter\def\csname PY@tok@s2\endcsname{\def\PY@tc##1{\textcolor[rgb]{0.73,0.13,0.13}{##1}}}
\expandafter\def\csname PY@tok@sh\endcsname{\def\PY@tc##1{\textcolor[rgb]{0.73,0.13,0.13}{##1}}}
\expandafter\def\csname PY@tok@s1\endcsname{\def\PY@tc##1{\textcolor[rgb]{0.73,0.13,0.13}{##1}}}
\expandafter\def\csname PY@tok@mb\endcsname{\def\PY@tc##1{\textcolor[rgb]{0.40,0.40,0.40}{##1}}}
\expandafter\def\csname PY@tok@mf\endcsname{\def\PY@tc##1{\textcolor[rgb]{0.40,0.40,0.40}{##1}}}
\expandafter\def\csname PY@tok@mh\endcsname{\def\PY@tc##1{\textcolor[rgb]{0.40,0.40,0.40}{##1}}}
\expandafter\def\csname PY@tok@mi\endcsname{\def\PY@tc##1{\textcolor[rgb]{0.40,0.40,0.40}{##1}}}
\expandafter\def\csname PY@tok@il\endcsname{\def\PY@tc##1{\textcolor[rgb]{0.40,0.40,0.40}{##1}}}
\expandafter\def\csname PY@tok@mo\endcsname{\def\PY@tc##1{\textcolor[rgb]{0.40,0.40,0.40}{##1}}}
\expandafter\def\csname PY@tok@ch\endcsname{\let\PY@it=\textit\def\PY@tc##1{\textcolor[rgb]{0.25,0.50,0.50}{##1}}}
\expandafter\def\csname PY@tok@cm\endcsname{\let\PY@it=\textit\def\PY@tc##1{\textcolor[rgb]{0.25,0.50,0.50}{##1}}}
\expandafter\def\csname PY@tok@cpf\endcsname{\let\PY@it=\textit\def\PY@tc##1{\textcolor[rgb]{0.25,0.50,0.50}{##1}}}
\expandafter\def\csname PY@tok@c1\endcsname{\let\PY@it=\textit\def\PY@tc##1{\textcolor[rgb]{0.25,0.50,0.50}{##1}}}
\expandafter\def\csname PY@tok@cs\endcsname{\let\PY@it=\textit\def\PY@tc##1{\textcolor[rgb]{0.25,0.50,0.50}{##1}}}

\def\PYZbs{\char`\\}
\def\PYZus{\char`\_}
\def\PYZob{\char`\{}
\def\PYZcb{\char`\}}
\def\PYZca{\char`\^}
\def\PYZam{\char`\&}
\def\PYZlt{\char`\<}
\def\PYZgt{\char`\>}
\def\PYZsh{\char`\#}
\def\PYZpc{\char`\%}
\def\PYZdl{\char`\$}
\def\PYZhy{\char`\-}
\def\PYZsq{\char`\'}
\def\PYZdq{\char`\"}
\def\PYZti{\char`\~}
% for compatibility with earlier versions
\def\PYZat{@}
\def\PYZlb{[}
\def\PYZrb{]}
\makeatother


    % Exact colors from NB
    \definecolor{incolor}{rgb}{0.0, 0.0, 0.5}
    \definecolor{outcolor}{rgb}{0.545, 0.0, 0.0}



    
    % Prevent overflowing lines due to hard-to-break entities
    \sloppy 
    % Setup hyperref package
    \hypersetup{
      breaklinks=true,  % so long urls are correctly broken across lines
      colorlinks=true,
      urlcolor=urlcolor,
      linkcolor=linkcolor,
      citecolor=citecolor,
      }
    % Slightly bigger margins than the latex defaults
    
    \geometry{verbose,tmargin=1in,bmargin=1in,lmargin=1in,rmargin=1in}
    
    

    \begin{document}
    
    
    \maketitle
    
    

    
    \section{Machine Learning Engineer
Nanodegree}\label{machine-learning-engineer-nanodegree}

\subsection{Unsupervised Learning}\label{unsupervised-learning}

\subsection{Project: Creating Customer
Segments}\label{project-creating-customer-segments}

    Welcome to the third project of the Machine Learning Engineer
Nanodegree! In this notebook, some template code has already been
provided for you, and it will be your job to implement the additional
functionality necessary to successfully complete this project. Sections
that begin with \textbf{'Implementation'} in the header indicate that
the following block of code will require additional functionality which
you must provide. Instructions will be provided for each section and the
specifics of the implementation are marked in the code block with a
\texttt{\textquotesingle{}TODO\textquotesingle{}} statement. Please be
sure to read the instructions carefully!

In addition to implementing code, there will be questions that you must
answer which relate to the project and your implementation. Each section
where you will answer a question is preceded by a \textbf{'Question X'}
header. Carefully read each question and provide thorough answers in the
following text boxes that begin with \textbf{'Answer:'}. Your project
submission will be evaluated based on your answers to each of the
questions and the implementation you provide.

\begin{quote}
\textbf{Note:} Code and Markdown cells can be executed using the
\textbf{Shift + Enter} keyboard shortcut. In addition, Markdown cells
can be edited by typically double-clicking the cell to enter edit mode.
\end{quote}

    \subsection{Getting Started}\label{getting-started}

In this project, you will analyze a dataset containing data on various
customers' annual spending amounts (reported in \emph{monetary units})
of diverse product categories for internal structure. One goal of this
project is to best describe the variation in the different types of
customers that a wholesale distributor interacts with. Doing so would
equip the distributor with insight into how to best structure their
delivery service to meet the needs of each customer.

The dataset for this project can be found on the
\href{https://archive.ics.uci.edu/ml/datasets/Wholesale+customers}{UCI
Machine Learning Repository}. For the purposes of this project, the
features \texttt{\textquotesingle{}Channel\textquotesingle{}} and
\texttt{\textquotesingle{}Region\textquotesingle{}} will be excluded in
the analysis --- with focus instead on the six product categories
recorded for customers.

Run the code block below to load the wholesale customers dataset, along
with a few of the necessary Python libraries required for this project.
You will know the dataset loaded successfully if the size of the dataset
is reported.

    \begin{Verbatim}[commandchars=\\\{\}]
{\color{incolor}In [{\color{incolor}194}]:} \PY{c+c1}{\PYZsh{} Import libraries necessary for this project}
          \PY{k+kn}{import} \PY{n+nn}{numpy} \PY{k}{as} \PY{n+nn}{np}
          \PY{k+kn}{import} \PY{n+nn}{pandas} \PY{k}{as} \PY{n+nn}{pd}
          \PY{k+kn}{from} \PY{n+nn}{IPython}\PY{n+nn}{.}\PY{n+nn}{display} \PY{k}{import} \PY{n}{display} \PY{c+c1}{\PYZsh{} Allows the use of display() for DataFrames}
          
          \PY{c+c1}{\PYZsh{} Import supplementary visualizations code visuals.py}
          \PY{k+kn}{import} \PY{n+nn}{visuals} \PY{k}{as} \PY{n+nn}{vs}
          
          \PY{c+c1}{\PYZsh{} Pretty display for notebooks}
          \PY{o}{\PYZpc{}}\PY{k}{matplotlib} inline
          
          \PY{c+c1}{\PYZsh{} Load the wholesale customers dataset}
          \PY{k}{try}\PY{p}{:}
              \PY{n}{data} \PY{o}{=} \PY{n}{pd}\PY{o}{.}\PY{n}{read\PYZus{}csv}\PY{p}{(}\PY{l+s+s2}{\PYZdq{}}\PY{l+s+s2}{customers.csv}\PY{l+s+s2}{\PYZdq{}}\PY{p}{)}
              \PY{n}{data}\PY{o}{.}\PY{n}{drop}\PY{p}{(}\PY{p}{[}\PY{l+s+s1}{\PYZsq{}}\PY{l+s+s1}{Region}\PY{l+s+s1}{\PYZsq{}}\PY{p}{,} \PY{l+s+s1}{\PYZsq{}}\PY{l+s+s1}{Channel}\PY{l+s+s1}{\PYZsq{}}\PY{p}{]}\PY{p}{,} \PY{n}{axis} \PY{o}{=} \PY{l+m+mi}{1}\PY{p}{,} \PY{n}{inplace} \PY{o}{=} \PY{k+kc}{True}\PY{p}{)}
              \PY{n+nb}{print}\PY{p}{(}\PY{l+s+s2}{\PYZdq{}}\PY{l+s+s2}{Wholesale customers dataset has }\PY{l+s+si}{\PYZob{}\PYZcb{}}\PY{l+s+s2}{ samples with }\PY{l+s+si}{\PYZob{}\PYZcb{}}\PY{l+s+s2}{ features each.}\PY{l+s+s2}{\PYZdq{}}\PY{o}{.}\PY{n}{format}\PY{p}{(}\PY{o}{*}\PY{n}{data}\PY{o}{.}\PY{n}{shape}\PY{p}{)}\PY{p}{)}
          \PY{k}{except}\PY{p}{:}
              \PY{n+nb}{print}\PY{p}{(}\PY{l+s+s2}{\PYZdq{}}\PY{l+s+s2}{Dataset could not be loaded. Is the dataset missing?}\PY{l+s+s2}{\PYZdq{}}\PY{p}{)}
\end{Verbatim}


    \begin{Verbatim}[commandchars=\\\{\}]
Wholesale customers dataset has 440 samples with 6 features each.

    \end{Verbatim}

    \subsection{Data Exploration}\label{data-exploration}

In this section, you will begin exploring the data through
visualizations and code to understand how each feature is related to the
others. You will observe a statistical description of the dataset,
consider the relevance of each feature, and select a few sample data
points from the dataset which you will track through the course of this
project.

Run the code block below to observe a statistical description of the
dataset. Note that the dataset is composed of six important product
categories: \textbf{'Fresh'}, \textbf{'Milk'}, \textbf{'Grocery'},
\textbf{'Frozen'}, \textbf{'Detergents\_Paper'}, and
\textbf{'Delicatessen'}. Consider what each category represents in terms
of products you could purchase.

    \begin{Verbatim}[commandchars=\\\{\}]
{\color{incolor}In [{\color{incolor}195}]:} \PY{c+c1}{\PYZsh{} Display a description of the dataset}
          \PY{n}{display}\PY{p}{(}\PY{n}{data}\PY{o}{.}\PY{n}{describe}\PY{p}{(}\PY{p}{)}\PY{p}{)}
\end{Verbatim}


    
    \begin{verbatim}
               Fresh          Milk       Grocery        Frozen  \
count     440.000000    440.000000    440.000000    440.000000   
mean    12000.297727   5796.265909   7951.277273   3071.931818   
std     12647.328865   7380.377175   9503.162829   4854.673333   
min         3.000000     55.000000      3.000000     25.000000   
25%      3127.750000   1533.000000   2153.000000    742.250000   
50%      8504.000000   3627.000000   4755.500000   1526.000000   
75%     16933.750000   7190.250000  10655.750000   3554.250000   
max    112151.000000  73498.000000  92780.000000  60869.000000   

       Detergents_Paper  Delicatessen  
count        440.000000    440.000000  
mean        2881.493182   1524.870455  
std         4767.854448   2820.105937  
min            3.000000      3.000000  
25%          256.750000    408.250000  
50%          816.500000    965.500000  
75%         3922.000000   1820.250000  
max        40827.000000  47943.000000  
    \end{verbatim}

    
    \subsubsection{Implementation: Selecting
Samples}\label{implementation-selecting-samples}

To get a better understanding of the customers and how their data will
transform through the analysis, it would be best to select a few sample
data points and explore them in more detail. In the code block below,
add \textbf{three} indices of your choice to the \texttt{indices} list
which will represent the customers to track. It is suggested to try
different sets of samples until you obtain customers that vary
significantly from one another.

    \begin{Verbatim}[commandchars=\\\{\}]
{\color{incolor}In [{\color{incolor}196}]:} \PY{c+c1}{\PYZsh{} TODO: Select three indices of your choice you wish to sample from the dataset}
          \PY{n}{indices} \PY{o}{=} \PY{p}{[}\PY{l+m+mi}{100}\PY{p}{,}\PY{l+m+mi}{150}\PY{p}{,}\PY{l+m+mi}{300}\PY{p}{]}
          
          \PY{c+c1}{\PYZsh{} Create a DataFrame of the chosen samples}
          \PY{n}{samples} \PY{o}{=} \PY{n}{pd}\PY{o}{.}\PY{n}{DataFrame}\PY{p}{(}\PY{n}{data}\PY{o}{.}\PY{n}{loc}\PY{p}{[}\PY{n}{indices}\PY{p}{]}\PY{p}{,} \PY{n}{columns} \PY{o}{=} \PY{n}{data}\PY{o}{.}\PY{n}{keys}\PY{p}{(}\PY{p}{)}\PY{p}{)}\PY{o}{.}\PY{n}{reset\PYZus{}index}\PY{p}{(}\PY{n}{drop} \PY{o}{=} \PY{k+kc}{True}\PY{p}{)}
          \PY{n+nb}{print}\PY{p}{(}\PY{l+s+s2}{\PYZdq{}}\PY{l+s+s2}{Chosen samples of wholesale customers dataset:}\PY{l+s+s2}{\PYZdq{}}\PY{p}{)}
          \PY{n}{display}\PY{p}{(}\PY{n}{samples}\PY{p}{)}
\end{Verbatim}


    \begin{Verbatim}[commandchars=\\\{\}]
Chosen samples of wholesale customers dataset:

    \end{Verbatim}

    
    \begin{verbatim}
   Fresh  Milk  Grocery  Frozen  Detergents_Paper  Delicatessen
0  11594  7779    12144    3252              8035          3029
1  16225  1825     1765     853               170          1067
2  16448  6243     6360     824              2662          2005
    \end{verbatim}

    
    \subsubsection{Question 1}\label{question-1}

Consider the total purchase cost of each product category and the
statistical description of the dataset above for your sample customers.

\begin{itemize}
\tightlist
\item
  What kind of establishment (customer) could each of the three samples
  you've chosen represent?
\end{itemize}

\textbf{Hint:} Examples of establishments include places like markets,
cafes, delis, wholesale retailers, among many others. Avoid using names
for establishments, such as saying \emph{"McDonalds"} when describing a
sample customer as a restaurant. You can use the mean values for
reference to compare your samples with. The mean values are as follows:

\begin{itemize}
\tightlist
\item
  Fresh: 12000.2977
\item
  Milk: 5796.2
\item
  Grocery: 7951.3
\item
  Detergents\_paper: 2881.4
\item
  Delicatessen: 1524.8
\end{itemize}

Knowing this, how do your samples compare? Does that help in driving
your insight into what kind of establishments they might be?

    \textbf{Answer:} For sample 0 All the products amount is above mean
value.Which means it is a super market or grocey shop. For sample 1 all
the product purchases are below mean value but Fresh cost is above to
mean.Which means the establishment is a flower shop. For sample 2 Fresh
and Delicatessen are higher than it's mean values except frozen
Remaining products are almost equal to their means so may be it comes
under restaurant.

    \subsubsection{Implementation: Feature
Relevance}\label{implementation-feature-relevance}

One interesting thought to consider is if one (or more) of the six
product categories is actually relevant for understanding customer
purchasing. That is to say, is it possible to determine whether
customers purchasing some amount of one category of products will
necessarily purchase some proportional amount of another category of
products? We can make this determination quite easily by training a
supervised regression learner on a subset of the data with one feature
removed, and then score how well that model can predict the removed
feature.

In the code block below, you will need to implement the following: -
Assign \texttt{new\_data} a copy of the data by removing a feature of
your choice using the \texttt{DataFrame.drop} function. - Use
\texttt{sklearn.cross\_validation.train\_test\_split} to split the
dataset into training and testing sets. - Use the removed feature as
your target label. Set a \texttt{test\_size} of \texttt{0.25} and set a
\texttt{random\_state}. - Import a decision tree regressor, set a
\texttt{random\_state}, and fit the learner to the training data. -
Report the prediction score of the testing set using the regressor's
\texttt{score} function.

    \begin{Verbatim}[commandchars=\\\{\}]
{\color{incolor}In [{\color{incolor}197}]:} \PY{c+c1}{\PYZsh{} TODO: Make a copy of the DataFrame, using the \PYZsq{}drop\PYZsq{} function to drop the given feature}
          \PY{n}{new\PYZus{}data} \PY{o}{=} \PY{n}{data}\PY{o}{.}\PY{n}{drop}\PY{p}{(}\PY{l+s+s1}{\PYZsq{}}\PY{l+s+s1}{Delicatessen}\PY{l+s+s1}{\PYZsq{}}\PY{p}{,}\PY{n}{axis}\PY{o}{=}\PY{l+m+mi}{1}\PY{p}{)}
          
          \PY{c+c1}{\PYZsh{} TODO: Split the data into training and testing sets(0.25) using the given feature as the target}
          \PY{c+c1}{\PYZsh{} Set a random state.}
          
          \PY{k+kn}{from} \PY{n+nn}{sklearn}\PY{n+nn}{.}\PY{n+nn}{cross\PYZus{}validation} \PY{k}{import} \PY{n}{train\PYZus{}test\PYZus{}split}
          \PY{n}{X\PYZus{}train}\PY{p}{,} \PY{n}{X\PYZus{}test}\PY{p}{,} \PY{n}{y\PYZus{}train}\PY{p}{,} \PY{n}{y\PYZus{}test} \PY{o}{=} \PY{n}{train\PYZus{}test\PYZus{}split}\PY{p}{(}\PY{n}{new\PYZus{}data}\PY{p}{,} \PY{n}{data}\PY{p}{[}\PY{l+s+s1}{\PYZsq{}}\PY{l+s+s1}{Delicatessen}\PY{l+s+s1}{\PYZsq{}}\PY{p}{]}\PY{p}{,} \PY{n}{test\PYZus{}size} \PY{o}{=} \PY{l+m+mf}{0.25}\PY{p}{,} \PY{n}{random\PYZus{}state}\PY{o}{=}\PY{l+m+mi}{42}\PY{p}{)}
          
          \PY{c+c1}{\PYZsh{} TODO: Create a decision tree regressor and fit it to the training set}
          \PY{k+kn}{from} \PY{n+nn}{sklearn}\PY{n+nn}{.}\PY{n+nn}{tree} \PY{k}{import} \PY{n}{DecisionTreeRegressor}
                                                             
          \PY{n}{regressor} \PY{o}{=} \PY{n}{DecisionTreeRegressor}\PY{p}{(}\PY{n}{random\PYZus{}state}\PY{o}{=}\PY{l+m+mi}{42}\PY{p}{)}\PY{o}{.}\PY{n}{fit}\PY{p}{(}\PY{n}{X\PYZus{}train}\PY{p}{,}\PY{n}{y\PYZus{}train}\PY{p}{)}
          
          \PY{c+c1}{\PYZsh{} TODO: Report the score of the prediction using the testing set}
          \PY{n}{score} \PY{o}{=} \PY{n}{regressor}\PY{o}{.}\PY{n}{score}\PY{p}{(}\PY{n}{X\PYZus{}test}\PY{p}{,}\PY{n}{y\PYZus{}test}\PY{p}{)}
          \PY{n+nb}{print}\PY{p}{(}\PY{n}{score}\PY{p}{)}
\end{Verbatim}


    \begin{Verbatim}[commandchars=\\\{\}]
-2.254711537203931

    \end{Verbatim}

    \subsubsection{Question 2}\label{question-2}

\begin{itemize}
\tightlist
\item
  Which feature did you attempt to predict?
\item
  What was the reported prediction score?
\item
  Is this feature necessary for identifying customers' spending habits?
\end{itemize}

\textbf{Hint:} The coefficient of determination, \texttt{R\^{}2}, is
scored between 0 and 1, with 1 being a perfect fit. A negative
\texttt{R\^{}2} implies the model fails to fit the data. If you get a
low score for a particular feature, that lends us to beleive that that
feature point is hard to predict using the other features, thereby
making it an important feature to consider when considering relevance.

    \textbf{Answer:} Here I chose Delicatessen as prediction
value.Prediction score is -2.2547.It is a negative R\^{}2 value which
indicates that this feature is hard tto predict from other features.It
is not related to other features but it is necessary to identifying
customer spending habits.

    \subsubsection{Visualize Feature
Distributions}\label{visualize-feature-distributions}

To get a better understanding of the dataset, we can construct a scatter
matrix of each of the six product features present in the data. If you
found that the feature you attempted to predict above is relevant for
identifying a specific customer, then the scatter matrix below may not
show any correlation between that feature and the others. Conversely, if
you believe that feature is not relevant for identifying a specific
customer, the scatter matrix might show a correlation between that
feature and another feature in the data. Run the code block below to
produce a scatter matrix.

    \begin{Verbatim}[commandchars=\\\{\}]
{\color{incolor}In [{\color{incolor}198}]:} \PY{c+c1}{\PYZsh{} Produce a scatter matrix for each pair of features in the data}
          \PY{n}{pd}\PY{o}{.}\PY{n}{scatter\PYZus{}matrix}\PY{p}{(}\PY{n}{data}\PY{p}{,} \PY{n}{alpha} \PY{o}{=} \PY{l+m+mf}{0.3}\PY{p}{,} \PY{n}{figsize} \PY{o}{=} \PY{p}{(}\PY{l+m+mi}{14}\PY{p}{,}\PY{l+m+mi}{8}\PY{p}{)}\PY{p}{,} \PY{n}{diagonal} \PY{o}{=} \PY{l+s+s1}{\PYZsq{}}\PY{l+s+s1}{kde}\PY{l+s+s1}{\PYZsq{}}\PY{p}{)}\PY{p}{;}
\end{Verbatim}


    \begin{Verbatim}[commandchars=\\\{\}]
C:\textbackslash{}Anaconda2\textbackslash{}envs\textbackslash{}tensorflowenv\textbackslash{}lib\textbackslash{}site-packages\textbackslash{}ipykernel\_launcher.py:2: FutureWarning: pandas.scatter\_matrix is deprecated. Use pandas.plotting.scatter\_matrix instead
  

    \end{Verbatim}

    \begin{center}
    \adjustimage{max size={0.9\linewidth}{0.9\paperheight}}{output_15_1.png}
    \end{center}
    { \hspace*{\fill} \\}
    
    \subsubsection{Question 3}\label{question-3}

\begin{itemize}
\tightlist
\item
  Using the scatter matrix as a reference, discuss the distribution of
  the dataset, specifically talk about the normality, outliers, large
  number of data points near 0 among others. If you need to sepearate
  out some of the plots individually to further accentuate your point,
  you may do so as well.
\item
  Are there any pairs of features which exhibit some degree of
  correlation?
\item
  Does this confirm or deny your suspicions about the relevance of the
  feature you attempted to predict?
\item
  How is the data for those features distributed?
\end{itemize}

\textbf{Hint:} Is the data normally distributed? Where do most of the
data points lie? You can use
\href{https://pandas.pydata.org/pandas-docs/stable/generated/pandas.DataFrame.corr.html}{corr()}
to get the feature correlations and then visualize them using a
\href{http://seaborn.pydata.org/generated/seaborn.heatmap.html}{heatmap}(the
data that would be fed into the heatmap would be the correlation values,
for eg: \texttt{data.corr()}) to gain further insight.

    \begin{Verbatim}[commandchars=\\\{\}]
{\color{incolor}In [{\color{incolor}199}]:} \PY{k+kn}{import} \PY{n+nn}{seaborn}
          \PY{n}{display}\PY{p}{(}\PY{n}{data}\PY{o}{.}\PY{n}{corr}\PY{p}{(}\PY{p}{)}\PY{p}{)}
          \PY{n}{seaborn}\PY{o}{.}\PY{n}{heatmap}\PY{p}{(}\PY{n}{data}\PY{o}{.}\PY{n}{corr}\PY{p}{(}\PY{p}{)}\PY{p}{)}
\end{Verbatim}


    
    \begin{verbatim}
                     Fresh      Milk   Grocery    Frozen  Detergents_Paper  \
Fresh             1.000000  0.100510 -0.011854  0.345881         -0.101953   
Milk              0.100510  1.000000  0.728335  0.123994          0.661816   
Grocery          -0.011854  0.728335  1.000000 -0.040193          0.924641   
Frozen            0.345881  0.123994 -0.040193  1.000000         -0.131525   
Detergents_Paper -0.101953  0.661816  0.924641 -0.131525          1.000000   
Delicatessen      0.244690  0.406368  0.205497  0.390947          0.069291   

                  Delicatessen  
Fresh                 0.244690  
Milk                  0.406368  
Grocery               0.205497  
Frozen                0.390947  
Detergents_Paper      0.069291  
Delicatessen          1.000000  
    \end{verbatim}

    
\begin{Verbatim}[commandchars=\\\{\}]
{\color{outcolor}Out[{\color{outcolor}199}]:} <matplotlib.axes.\_subplots.AxesSubplot at 0x29a3e27ecc0>
\end{Verbatim}
            
    \begin{center}
    \adjustimage{max size={0.9\linewidth}{0.9\paperheight}}{output_17_2.png}
    \end{center}
    { \hspace*{\fill} \\}
    
    \textbf{Answer:} From the scatter plot above it is not normally
distrubuted.Most of the cases the plot is skewed right. In some cases
there are some outliers.Example In case of milk ang grocery there are
some outliers There are some features which are exibiting some degree of
correlation. 1.Grocery \& Detergents\_Paper have max correlation of
0.924641. 2.Milk and Grocery have 0.728335 correlation value 3.Milk and
Detergents\_Pape have 0.661816 Remaining all are having correlation
value less than 0.5 and negative values. My suspician is correct in case
of Delicatessen it is not exactly correlted with any value. Some of the
combination other than above combinations like Grocery and fresh are
near 0.

    \subsection{Data Preprocessing}\label{data-preprocessing}

In this section, you will preprocess the data to create a better
representation of customers by performing a scaling on the data and
detecting (and optionally removing) outliers. Preprocessing data is
often times a critical step in assuring that results you obtain from
your analysis are significant and meaningful.

    \subsubsection{Implementation: Feature
Scaling}\label{implementation-feature-scaling}

If data is not normally distributed, especially if the mean and median
vary significantly (indicating a large skew), it is most
\href{http://econbrowser.com/archives/2014/02/use-of-logarithms-in-economics}{often
appropriate} to apply a non-linear scaling --- particularly for
financial data. One way to achieve this scaling is by using a
\href{http://scipy.github.io/devdocs/generated/scipy.stats.boxcox.html}{Box-Cox
test}, which calculates the best power transformation of the data that
reduces skewness. A simpler approach which can work in most cases would
be applying the natural logarithm.

In the code block below, you will need to implement the following: -
Assign a copy of the data to \texttt{log\_data} after applying
logarithmic scaling. Use the \texttt{np.log} function for this. - Assign
a copy of the sample data to \texttt{log\_samples} after applying
logarithmic scaling. Again, use \texttt{np.log}.

    \begin{Verbatim}[commandchars=\\\{\}]
{\color{incolor}In [{\color{incolor}200}]:} \PY{c+c1}{\PYZsh{} TODO: Scale the data using the natural logarithm}
          \PY{n}{log\PYZus{}data} \PY{o}{=} \PY{n}{np}\PY{o}{.}\PY{n}{log}\PY{p}{(}\PY{n}{data}\PY{p}{)}
          
          \PY{c+c1}{\PYZsh{} TODO: Scale the sample data using the natural logarithm}
          \PY{n}{log\PYZus{}samples} \PY{o}{=} \PY{n}{np}\PY{o}{.}\PY{n}{log}\PY{p}{(}\PY{n}{samples}\PY{p}{)}
          
          \PY{c+c1}{\PYZsh{} Produce a scatter matrix for each pair of newly\PYZhy{}transformed features}
          \PY{n}{pd}\PY{o}{.}\PY{n}{scatter\PYZus{}matrix}\PY{p}{(}\PY{n}{log\PYZus{}data}\PY{p}{,} \PY{n}{alpha} \PY{o}{=} \PY{l+m+mf}{0.3}\PY{p}{,} \PY{n}{figsize} \PY{o}{=} \PY{p}{(}\PY{l+m+mi}{14}\PY{p}{,}\PY{l+m+mi}{8}\PY{p}{)}\PY{p}{,} \PY{n}{diagonal} \PY{o}{=} \PY{l+s+s1}{\PYZsq{}}\PY{l+s+s1}{kde}\PY{l+s+s1}{\PYZsq{}}\PY{p}{)}\PY{p}{;}
\end{Verbatim}


    \begin{Verbatim}[commandchars=\\\{\}]
C:\textbackslash{}Anaconda2\textbackslash{}envs\textbackslash{}tensorflowenv\textbackslash{}lib\textbackslash{}site-packages\textbackslash{}ipykernel\_launcher.py:8: FutureWarning: pandas.scatter\_matrix is deprecated. Use pandas.plotting.scatter\_matrix instead
  

    \end{Verbatim}

    \begin{center}
    \adjustimage{max size={0.9\linewidth}{0.9\paperheight}}{output_21_1.png}
    \end{center}
    { \hspace*{\fill} \\}
    
    \begin{Verbatim}[commandchars=\\\{\}]
{\color{incolor}In [{\color{incolor}201}]:} \PY{n}{display}\PY{p}{(}\PY{n}{log\PYZus{}data}\PY{o}{.}\PY{n}{corr}\PY{p}{(}\PY{p}{)}\PY{p}{)}
          \PY{n}{seaborn}\PY{o}{.}\PY{n}{heatmap}\PY{p}{(}\PY{n}{log\PYZus{}data}\PY{o}{.}\PY{n}{corr}\PY{p}{(}\PY{p}{)}\PY{p}{)}
\end{Verbatim}


    
    \begin{verbatim}
                     Fresh      Milk   Grocery    Frozen  Detergents_Paper  \
Fresh             1.000000 -0.019834 -0.132713  0.383996         -0.155871   
Milk             -0.019834  1.000000  0.758851 -0.055316          0.677942   
Grocery          -0.132713  0.758851  1.000000 -0.164524          0.796398   
Frozen            0.383996 -0.055316 -0.164524  1.000000         -0.211576   
Detergents_Paper -0.155871  0.677942  0.796398 -0.211576          1.000000   
Delicatessen      0.255186  0.337833  0.235728  0.254718          0.166735   

                  Delicatessen  
Fresh                 0.255186  
Milk                  0.337833  
Grocery               0.235728  
Frozen                0.254718  
Detergents_Paper      0.166735  
Delicatessen          1.000000  
    \end{verbatim}

    
\begin{Verbatim}[commandchars=\\\{\}]
{\color{outcolor}Out[{\color{outcolor}201}]:} <matplotlib.axes.\_subplots.AxesSubplot at 0x29a3da4ec18>
\end{Verbatim}
            
    \begin{center}
    \adjustimage{max size={0.9\linewidth}{0.9\paperheight}}{output_22_2.png}
    \end{center}
    { \hspace*{\fill} \\}
    
    \subsubsection{Observation}\label{observation}

After applying a natural logarithm scaling to the data, the distribution
of each feature should appear much more normal. For any pairs of
features you may have identified earlier as being correlated, observe
here whether that correlation is still present (and whether it is now
stronger or weaker than before).

Run the code below to see how the sample data has changed after having
the natural logarithm applied to it.

    \begin{Verbatim}[commandchars=\\\{\}]
{\color{incolor}In [{\color{incolor}202}]:} \PY{c+c1}{\PYZsh{} Display the log\PYZhy{}transformed sample data}
          \PY{n}{display}\PY{p}{(}\PY{n}{log\PYZus{}samples}\PY{p}{)}
\end{Verbatim}


    
    \begin{verbatim}
      Fresh      Milk   Grocery    Frozen  Detergents_Paper  Delicatessen
0  9.358243  8.959183  9.404590  8.087025          8.991562      8.015988
1  9.694309  7.509335  7.475906  6.748760          5.135798      6.972606
2  9.707959  8.739216  8.757784  6.714171          7.886833      7.603399
    \end{verbatim}

    
    \subsubsection{Implementation: Outlier
Detection}\label{implementation-outlier-detection}

Detecting outliers in the data is extremely important in the data
preprocessing step of any analysis. The presence of outliers can often
skew results which take into consideration these data points. There are
many "rules of thumb" for what constitutes an outlier in a dataset.
Here, we will use
\href{http://datapigtechnologies.com/blog/index.php/highlighting-outliers-in-your-data-with-the-tukey-method/}{Tukey's
Method for identfying outliers}: An \emph{outlier step} is calculated as
1.5 times the interquartile range (IQR). A data point with a feature
that is beyond an outlier step outside of the IQR for that feature is
considered abnormal.

In the code block below, you will need to implement the following: -
Assign the value of the 25th percentile for the given feature to
\texttt{Q1}. Use \texttt{np.percentile} for this. - Assign the value of
the 75th percentile for the given feature to \texttt{Q3}. Again, use
\texttt{np.percentile}. - Assign the calculation of an outlier step for
the given feature to \texttt{step}. - Optionally remove data points from
the dataset by adding indices to the \texttt{outliers} list.

\textbf{NOTE:} If you choose to remove any outliers, ensure that the
sample data does not contain any of these points!\\
Once you have performed this implementation, the dataset will be stored
in the variable \texttt{good\_data}.

    \begin{Verbatim}[commandchars=\\\{\}]
{\color{incolor}In [{\color{incolor}203}]:} \PY{c+c1}{\PYZsh{} For each feature find the data points with extreme high or low values}
          \PY{k}{for} \PY{n}{feature} \PY{o+ow}{in} \PY{n}{log\PYZus{}data}\PY{o}{.}\PY{n}{keys}\PY{p}{(}\PY{p}{)}\PY{p}{:}
              
              \PY{c+c1}{\PYZsh{} TODO: Calculate Q1 (25th percentile of the data) for the given feature}
              \PY{n}{Q1} \PY{o}{=} \PY{n}{np}\PY{o}{.}\PY{n}{percentile}\PY{p}{(}\PY{n}{log\PYZus{}data}\PY{p}{[}\PY{n}{feature}\PY{p}{]}\PY{p}{,}\PY{l+m+mf}{0.25}\PY{p}{)}
              
              \PY{c+c1}{\PYZsh{} TODO: Calculate Q3 (75th percentile of the data) for the given feature}
              \PY{n}{Q3} \PY{o}{=} \PY{n}{np}\PY{o}{.}\PY{n}{percentile}\PY{p}{(}\PY{n}{log\PYZus{}data}\PY{p}{[}\PY{n}{feature}\PY{p}{]}\PY{p}{,}\PY{l+m+mf}{0.75}\PY{p}{)}
              
              \PY{c+c1}{\PYZsh{} TODO: Use the interquartile range to calculate an outlier step (1.5 times the interquartile range)}
              \PY{n}{step} \PY{o}{=} \PY{l+m+mf}{1.5}\PY{o}{*}\PY{p}{(}\PY{n}{Q3}\PY{o}{\PYZhy{}}\PY{n}{Q1}\PY{p}{)}
              
              \PY{c+c1}{\PYZsh{} Display the outliers}
              \PY{n+nb}{print}\PY{p}{(}\PY{l+s+s2}{\PYZdq{}}\PY{l+s+s2}{Data points considered outliers for the feature }\PY{l+s+s2}{\PYZsq{}}\PY{l+s+si}{\PYZob{}\PYZcb{}}\PY{l+s+s2}{\PYZsq{}}\PY{l+s+s2}{:}\PY{l+s+s2}{\PYZdq{}}\PY{o}{.}\PY{n}{format}\PY{p}{(}\PY{n}{feature}\PY{p}{)}\PY{p}{)}
              \PY{c+c1}{\PYZsh{}display(log\PYZus{}data[\PYZti{}((log\PYZus{}data[feature] \PYZgt{}= Q1 \PYZhy{} step) \PYZam{} (log\PYZus{}data[feature] \PYZlt{}= Q3 + step))])}
              \PY{n+nb}{print}\PY{p}{(}\PY{n}{log\PYZus{}data}\PY{p}{[}\PY{o}{\PYZti{}}\PY{p}{(}\PY{p}{(}\PY{n}{log\PYZus{}data}\PY{p}{[}\PY{n}{feature}\PY{p}{]} \PY{o}{\PYZgt{}}\PY{o}{=} \PY{n}{Q1} \PY{o}{\PYZhy{}} \PY{n}{step}\PY{p}{)} \PY{o}{\PYZam{}} \PY{p}{(}\PY{n}{log\PYZus{}data}\PY{p}{[}\PY{n}{feature}\PY{p}{]} \PY{o}{\PYZlt{}}\PY{o}{=} \PY{n}{Q3} \PY{o}{+} \PY{n}{step}\PY{p}{)}\PY{p}{)}\PY{p}{]}\PY{p}{)}
              
          \PY{c+c1}{\PYZsh{} OPTIONAL: Select the indices for data points you wish to remove}
          \PY{n}{outliers}\PY{o}{=}\PY{p}{[}\PY{l+m+mi}{65}\PY{p}{,}\PY{l+m+mi}{66}\PY{p}{,}\PY{l+m+mi}{75}\PY{p}{,}\PY{l+m+mi}{128}\PY{p}{,}\PY{l+m+mi}{154}\PY{p}{]}
          
          \PY{c+c1}{\PYZsh{} Remove the outliers, if any were specified}
          \PY{n}{good\PYZus{}data} \PY{o}{=} \PY{n}{log\PYZus{}data}\PY{o}{.}\PY{n}{drop}\PY{p}{(}\PY{n}{log\PYZus{}data}\PY{o}{.}\PY{n}{index}\PY{p}{[}\PY{n}{outliers}\PY{p}{]}\PY{p}{)}\PY{o}{.}\PY{n}{reset\PYZus{}index}\PY{p}{(}\PY{n}{drop} \PY{o}{=} \PY{k+kc}{True}\PY{p}{)}
\end{Verbatim}


    \begin{Verbatim}[commandchars=\\\{\}]
Data points considered outliers for the feature 'Fresh':
         Fresh       Milk    Grocery    Frozen  Detergents\_Paper  Delicatessen
0     9.446913   9.175335   8.930759  5.365976          7.891331      7.198931
1     8.861775   9.191158   9.166179  7.474205          8.099554      7.482119
2     8.756682   9.083416   8.946896  7.785305          8.165079      8.967504
3     9.492884   7.086738   8.347827  8.764678          6.228511      7.488853
4    10.026369   8.596004   8.881558  8.272571          7.482682      8.553525
5     9.149847   9.019059   8.542081  6.501290          7.492760      7.280008
6     9.403107   8.070594   8.850088  6.173786          8.051978      6.300786
7     8.933137   8.508354   9.151227  7.419980          8.108021      7.850104
8     8.693329   8.201934   8.731013  6.052089          7.447751      6.620073
9     8.700514   9.314070   9.845911  7.055313          8.912608      7.648740
10    8.121480   8.594710   9.470703  8.389360          8.695674      7.463937
11    9.483873   7.024649   8.416931  7.258412          6.308098      6.208590
12   10.364514   9.418898   9.372204  5.659482          8.263848      7.983099
13    9.962558   8.733594   9.614605  8.037543          8.810907      6.400257
14   10.112654   9.155356   9.400217  5.683580          8.528726      7.681560
15    9.235326   7.015712   8.248267  5.983936          6.871091      6.021023
16    6.927558   9.084324   9.402695  4.897840          8.413609      6.984716
17    8.678632   8.725345   7.983781  6.732211          5.913503      8.406932
18    9.830971   8.752581   9.220192  7.698483          7.925519      8.064951
19    8.959312   7.822044   9.155250  6.505784          7.831220      6.216606
20    9.772581   8.416046   8.434246  6.971669          7.722678      7.661056
21    8.624612   6.769642   7.605890  8.126518          5.926926      6.343880
22   10.350606   7.558517   8.404920  9.149316          7.775276      8.374246
23   10.180096  10.502956   9.999661  8.547528          8.374938      9.712509
24   10.027783   9.187686   9.531844  7.977625          8.407825      8.661813
25    9.690604   8.349957   8.935245  5.303305          8.294799      4.043051
26    9.200088   6.867974   7.958926  8.055475          5.488938      6.725034
27    9.566335   6.688355   8.021256  6.184149          4.605170      6.249975
28    8.321908   9.927399  10.164197  7.054450          9.059982      8.557567
29   10.671000   7.649693   7.866722  7.090077          7.009409      6.712956
..         {\ldots}        {\ldots}        {\ldots}       {\ldots}               {\ldots}           {\ldots}
409   9.071997   8.198089   8.716044  7.761745          7.660585      8.544225
410   8.799812   7.647786   8.425736  7.236339          7.528332      7.545390
411   7.661998   8.098339   8.095904  7.336286          5.459586      8.381373
413   8.513787   8.488588   8.799812  9.790655          6.815640      7.797702
414   8.694335   7.595890   8.136518  8.644530          7.034388      5.669881
415   8.967249   8.707152   9.053920  7.433075          8.171882      7.535830
416   8.386857   9.300181   9.297252  6.742881          8.814033      6.900731
417   8.530109   8.612322   9.310638  5.897154          8.156223      6.968850
418   6.492240   9.047115   9.832099  4.890349          8.815815      6.654153
419   9.089415   8.238273   7.706613  6.450470          7.365180      7.327123
420   8.402007   8.569026   9.490015  3.218876          8.827321      7.239215
421   9.744668   8.486115   9.110851  6.938284          8.135933      7.486613
422  10.181119   7.227662   8.336151  6.721426          6.854355      7.104965
423   9.773664   8.212297   8.446127  6.965080          7.497207      6.504288
424   9.739791   7.966933   9.411811  6.773080          8.074960      5.517453
425   9.327501   7.786552   7.860571  9.638740          4.682131      7.542213
426   9.482960   9.142811   9.569133  8.052296          8.532870      7.546446
427  10.342130   9.722385   8.599510  9.621257          6.084499      7.058758
428   8.021913   8.694502   8.499029  7.695303          6.745236      5.758902
429   9.060331   7.467371   8.183118  3.850148          4.430817      7.824446
430   8.038189   8.349957   9.710085  6.354370          5.484797      7.640123
431   9.051696   8.613594   8.548692  9.509407          7.227662      7.311886
432   9.957834   7.057898   8.466742  5.594711          7.191429      5.978886
433   7.591862   8.076515   7.308543  7.340187          5.874931      7.278629
434   9.725019   8.274357   8.986447  6.533789          7.771067      6.731018
435  10.299003   9.396903   9.682030  9.483036          5.204007      7.698029
436  10.577146   7.266129   6.638568  8.414052          4.532599      7.760467
437   9.584040   9.647821  10.317020  6.079933          9.605149      7.532088
438   9.238928   7.591357   7.710653  6.945051          5.123964      7.661527
439   7.932721   7.437206   7.828038  4.174387          6.167516      3.951244

[422 rows x 6 columns]
Data points considered outliers for the feature 'Milk':
         Fresh       Milk    Grocery    Frozen  Detergents\_Paper  Delicatessen
0     9.446913   9.175335   8.930759  5.365976          7.891331      7.198931
1     8.861775   9.191158   9.166179  7.474205          8.099554      7.482119
2     8.756682   9.083416   8.946896  7.785305          8.165079      8.967504
3     9.492884   7.086738   8.347827  8.764678          6.228511      7.488853
4    10.026369   8.596004   8.881558  8.272571          7.482682      8.553525
5     9.149847   9.019059   8.542081  6.501290          7.492760      7.280008
6     9.403107   8.070594   8.850088  6.173786          8.051978      6.300786
7     8.933137   8.508354   9.151227  7.419980          8.108021      7.850104
8     8.693329   8.201934   8.731013  6.052089          7.447751      6.620073
9     8.700514   9.314070   9.845911  7.055313          8.912608      7.648740
10    8.121480   8.594710   9.470703  8.389360          8.695674      7.463937
11    9.483873   7.024649   8.416931  7.258412          6.308098      6.208590
12   10.364514   9.418898   9.372204  5.659482          8.263848      7.983099
13    9.962558   8.733594   9.614605  8.037543          8.810907      6.400257
14   10.112654   9.155356   9.400217  5.683580          8.528726      7.681560
15    9.235326   7.015712   8.248267  5.983936          6.871091      6.021023
16    6.927558   9.084324   9.402695  4.897840          8.413609      6.984716
17    8.678632   8.725345   7.983781  6.732211          5.913503      8.406932
18    9.830971   8.752581   9.220192  7.698483          7.925519      8.064951
19    8.959312   7.822044   9.155250  6.505784          7.831220      6.216606
20    9.772581   8.416046   8.434246  6.971669          7.722678      7.661056
21    8.624612   6.769642   7.605890  8.126518          5.926926      6.343880
22   10.350606   7.558517   8.404920  9.149316          7.775276      8.374246
23   10.180096  10.502956   9.999661  8.547528          8.374938      9.712509
24   10.027783   9.187686   9.531844  7.977625          8.407825      8.661813
25    9.690604   8.349957   8.935245  5.303305          8.294799      4.043051
26    9.200088   6.867974   7.958926  8.055475          5.488938      6.725034
27    9.566335   6.688355   8.021256  6.184149          4.605170      6.249975
28    8.321908   9.927399  10.164197  7.054450          9.059982      8.557567
29   10.671000   7.649693   7.866722  7.090077          7.009409      6.712956
..         {\ldots}        {\ldots}        {\ldots}       {\ldots}               {\ldots}           {\ldots}
410   8.799812   7.647786   8.425736  7.236339          7.528332      7.545390
411   7.661998   8.098339   8.095904  7.336286          5.459586      8.381373
412   4.574711   8.190077   9.425452  4.584967          7.996317      4.127134
413   8.513787   8.488588   8.799812  9.790655          6.815640      7.797702
414   8.694335   7.595890   8.136518  8.644530          7.034388      5.669881
415   8.967249   8.707152   9.053920  7.433075          8.171882      7.535830
416   8.386857   9.300181   9.297252  6.742881          8.814033      6.900731
417   8.530109   8.612322   9.310638  5.897154          8.156223      6.968850
418   6.492240   9.047115   9.832099  4.890349          8.815815      6.654153
419   9.089415   8.238273   7.706613  6.450470          7.365180      7.327123
420   8.402007   8.569026   9.490015  3.218876          8.827321      7.239215
421   9.744668   8.486115   9.110851  6.938284          8.135933      7.486613
422  10.181119   7.227662   8.336151  6.721426          6.854355      7.104965
423   9.773664   8.212297   8.446127  6.965080          7.497207      6.504288
424   9.739791   7.966933   9.411811  6.773080          8.074960      5.517453
425   9.327501   7.786552   7.860571  9.638740          4.682131      7.542213
426   9.482960   9.142811   9.569133  8.052296          8.532870      7.546446
427  10.342130   9.722385   8.599510  9.621257          6.084499      7.058758
428   8.021913   8.694502   8.499029  7.695303          6.745236      5.758902
429   9.060331   7.467371   8.183118  3.850148          4.430817      7.824446
430   8.038189   8.349957   9.710085  6.354370          5.484797      7.640123
431   9.051696   8.613594   8.548692  9.509407          7.227662      7.311886
432   9.957834   7.057898   8.466742  5.594711          7.191429      5.978886
433   7.591862   8.076515   7.308543  7.340187          5.874931      7.278629
434   9.725019   8.274357   8.986447  6.533789          7.771067      6.731018
435  10.299003   9.396903   9.682030  9.483036          5.204007      7.698029
436  10.577146   7.266129   6.638568  8.414052          4.532599      7.760467
437   9.584040   9.647821  10.317020  6.079933          9.605149      7.532088
438   9.238928   7.591357   7.710653  6.945051          5.123964      7.661527
439   7.932721   7.437206   7.828038  4.174387          6.167516      3.951244

[422 rows x 6 columns]
Data points considered outliers for the feature 'Grocery':
         Fresh       Milk    Grocery    Frozen  Detergents\_Paper  Delicatessen
0     9.446913   9.175335   8.930759  5.365976          7.891331      7.198931
1     8.861775   9.191158   9.166179  7.474205          8.099554      7.482119
2     8.756682   9.083416   8.946896  7.785305          8.165079      8.967504
3     9.492884   7.086738   8.347827  8.764678          6.228511      7.488853
4    10.026369   8.596004   8.881558  8.272571          7.482682      8.553525
5     9.149847   9.019059   8.542081  6.501290          7.492760      7.280008
6     9.403107   8.070594   8.850088  6.173786          8.051978      6.300786
7     8.933137   8.508354   9.151227  7.419980          8.108021      7.850104
8     8.693329   8.201934   8.731013  6.052089          7.447751      6.620073
9     8.700514   9.314070   9.845911  7.055313          8.912608      7.648740
10    8.121480   8.594710   9.470703  8.389360          8.695674      7.463937
11    9.483873   7.024649   8.416931  7.258412          6.308098      6.208590
12   10.364514   9.418898   9.372204  5.659482          8.263848      7.983099
13    9.962558   8.733594   9.614605  8.037543          8.810907      6.400257
14   10.112654   9.155356   9.400217  5.683580          8.528726      7.681560
15    9.235326   7.015712   8.248267  5.983936          6.871091      6.021023
16    6.927558   9.084324   9.402695  4.897840          8.413609      6.984716
17    8.678632   8.725345   7.983781  6.732211          5.913503      8.406932
18    9.830971   8.752581   9.220192  7.698483          7.925519      8.064951
19    8.959312   7.822044   9.155250  6.505784          7.831220      6.216606
20    9.772581   8.416046   8.434246  6.971669          7.722678      7.661056
21    8.624612   6.769642   7.605890  8.126518          5.926926      6.343880
22   10.350606   7.558517   8.404920  9.149316          7.775276      8.374246
23   10.180096  10.502956   9.999661  8.547528          8.374938      9.712509
24   10.027783   9.187686   9.531844  7.977625          8.407825      8.661813
25    9.690604   8.349957   8.935245  5.303305          8.294799      4.043051
26    9.200088   6.867974   7.958926  8.055475          5.488938      6.725034
27    9.566335   6.688355   8.021256  6.184149          4.605170      6.249975
28    8.321908   9.927399  10.164197  7.054450          9.059982      8.557567
29   10.671000   7.649693   7.866722  7.090077          7.009409      6.712956
..         {\ldots}        {\ldots}        {\ldots}       {\ldots}               {\ldots}           {\ldots}
410   8.799812   7.647786   8.425736  7.236339          7.528332      7.545390
411   7.661998   8.098339   8.095904  7.336286          5.459586      8.381373
412   4.574711   8.190077   9.425452  4.584967          7.996317      4.127134
413   8.513787   8.488588   8.799812  9.790655          6.815640      7.797702
414   8.694335   7.595890   8.136518  8.644530          7.034388      5.669881
415   8.967249   8.707152   9.053920  7.433075          8.171882      7.535830
416   8.386857   9.300181   9.297252  6.742881          8.814033      6.900731
417   8.530109   8.612322   9.310638  5.897154          8.156223      6.968850
418   6.492240   9.047115   9.832099  4.890349          8.815815      6.654153
419   9.089415   8.238273   7.706613  6.450470          7.365180      7.327123
420   8.402007   8.569026   9.490015  3.218876          8.827321      7.239215
421   9.744668   8.486115   9.110851  6.938284          8.135933      7.486613
422  10.181119   7.227662   8.336151  6.721426          6.854355      7.104965
423   9.773664   8.212297   8.446127  6.965080          7.497207      6.504288
424   9.739791   7.966933   9.411811  6.773080          8.074960      5.517453
425   9.327501   7.786552   7.860571  9.638740          4.682131      7.542213
426   9.482960   9.142811   9.569133  8.052296          8.532870      7.546446
427  10.342130   9.722385   8.599510  9.621257          6.084499      7.058758
428   8.021913   8.694502   8.499029  7.695303          6.745236      5.758902
429   9.060331   7.467371   8.183118  3.850148          4.430817      7.824446
430   8.038189   8.349957   9.710085  6.354370          5.484797      7.640123
431   9.051696   8.613594   8.548692  9.509407          7.227662      7.311886
432   9.957834   7.057898   8.466742  5.594711          7.191429      5.978886
433   7.591862   8.076515   7.308543  7.340187          5.874931      7.278629
434   9.725019   8.274357   8.986447  6.533789          7.771067      6.731018
435  10.299003   9.396903   9.682030  9.483036          5.204007      7.698029
436  10.577146   7.266129   6.638568  8.414052          4.532599      7.760467
437   9.584040   9.647821  10.317020  6.079933          9.605149      7.532088
438   9.238928   7.591357   7.710653  6.945051          5.123964      7.661527
439   7.932721   7.437206   7.828038  4.174387          6.167516      3.951244

[434 rows x 6 columns]
Data points considered outliers for the feature 'Frozen':
         Fresh       Milk    Grocery    Frozen  Detergents\_Paper  Delicatessen
0     9.446913   9.175335   8.930759  5.365976          7.891331      7.198931
1     8.861775   9.191158   9.166179  7.474205          8.099554      7.482119
2     8.756682   9.083416   8.946896  7.785305          8.165079      8.967504
3     9.492884   7.086738   8.347827  8.764678          6.228511      7.488853
4    10.026369   8.596004   8.881558  8.272571          7.482682      8.553525
5     9.149847   9.019059   8.542081  6.501290          7.492760      7.280008
6     9.403107   8.070594   8.850088  6.173786          8.051978      6.300786
7     8.933137   8.508354   9.151227  7.419980          8.108021      7.850104
8     8.693329   8.201934   8.731013  6.052089          7.447751      6.620073
9     8.700514   9.314070   9.845911  7.055313          8.912608      7.648740
10    8.121480   8.594710   9.470703  8.389360          8.695674      7.463937
11    9.483873   7.024649   8.416931  7.258412          6.308098      6.208590
12   10.364514   9.418898   9.372204  5.659482          8.263848      7.983099
13    9.962558   8.733594   9.614605  8.037543          8.810907      6.400257
14   10.112654   9.155356   9.400217  5.683580          8.528726      7.681560
15    9.235326   7.015712   8.248267  5.983936          6.871091      6.021023
16    6.927558   9.084324   9.402695  4.897840          8.413609      6.984716
17    8.678632   8.725345   7.983781  6.732211          5.913503      8.406932
18    9.830971   8.752581   9.220192  7.698483          7.925519      8.064951
19    8.959312   7.822044   9.155250  6.505784          7.831220      6.216606
20    9.772581   8.416046   8.434246  6.971669          7.722678      7.661056
21    8.624612   6.769642   7.605890  8.126518          5.926926      6.343880
22   10.350606   7.558517   8.404920  9.149316          7.775276      8.374246
23   10.180096  10.502956   9.999661  8.547528          8.374938      9.712509
24   10.027783   9.187686   9.531844  7.977625          8.407825      8.661813
25    9.690604   8.349957   8.935245  5.303305          8.294799      4.043051
26    9.200088   6.867974   7.958926  8.055475          5.488938      6.725034
27    9.566335   6.688355   8.021256  6.184149          4.605170      6.249975
28    8.321908   9.927399  10.164197  7.054450          9.059982      8.557567
29   10.671000   7.649693   7.866722  7.090077          7.009409      6.712956
..         {\ldots}        {\ldots}        {\ldots}       {\ldots}               {\ldots}           {\ldots}
409   9.071997   8.198089   8.716044  7.761745          7.660585      8.544225
410   8.799812   7.647786   8.425736  7.236339          7.528332      7.545390
411   7.661998   8.098339   8.095904  7.336286          5.459586      8.381373
412   4.574711   8.190077   9.425452  4.584967          7.996317      4.127134
413   8.513787   8.488588   8.799812  9.790655          6.815640      7.797702
414   8.694335   7.595890   8.136518  8.644530          7.034388      5.669881
415   8.967249   8.707152   9.053920  7.433075          8.171882      7.535830
416   8.386857   9.300181   9.297252  6.742881          8.814033      6.900731
417   8.530109   8.612322   9.310638  5.897154          8.156223      6.968850
418   6.492240   9.047115   9.832099  4.890349          8.815815      6.654153
419   9.089415   8.238273   7.706613  6.450470          7.365180      7.327123
420   8.402007   8.569026   9.490015  3.218876          8.827321      7.239215
421   9.744668   8.486115   9.110851  6.938284          8.135933      7.486613
422  10.181119   7.227662   8.336151  6.721426          6.854355      7.104965
423   9.773664   8.212297   8.446127  6.965080          7.497207      6.504288
424   9.739791   7.966933   9.411811  6.773080          8.074960      5.517453
425   9.327501   7.786552   7.860571  9.638740          4.682131      7.542213
426   9.482960   9.142811   9.569133  8.052296          8.532870      7.546446
427  10.342130   9.722385   8.599510  9.621257          6.084499      7.058758
428   8.021913   8.694502   8.499029  7.695303          6.745236      5.758902
430   8.038189   8.349957   9.710085  6.354370          5.484797      7.640123
431   9.051696   8.613594   8.548692  9.509407          7.227662      7.311886
432   9.957834   7.057898   8.466742  5.594711          7.191429      5.978886
433   7.591862   8.076515   7.308543  7.340187          5.874931      7.278629
434   9.725019   8.274357   8.986447  6.533789          7.771067      6.731018
435  10.299003   9.396903   9.682030  9.483036          5.204007      7.698029
436  10.577146   7.266129   6.638568  8.414052          4.532599      7.760467
437   9.584040   9.647821  10.317020  6.079933          9.605149      7.532088
438   9.238928   7.591357   7.710653  6.945051          5.123964      7.661527
439   7.932721   7.437206   7.828038  4.174387          6.167516      3.951244

[435 rows x 6 columns]
Data points considered outliers for the feature 'Detergents\_Paper':
         Fresh       Milk    Grocery    Frozen  Detergents\_Paper  Delicatessen
0     9.446913   9.175335   8.930759  5.365976          7.891331      7.198931
1     8.861775   9.191158   9.166179  7.474205          8.099554      7.482119
2     8.756682   9.083416   8.946896  7.785305          8.165079      8.967504
3     9.492884   7.086738   8.347827  8.764678          6.228511      7.488853
4    10.026369   8.596004   8.881558  8.272571          7.482682      8.553525
5     9.149847   9.019059   8.542081  6.501290          7.492760      7.280008
6     9.403107   8.070594   8.850088  6.173786          8.051978      6.300786
7     8.933137   8.508354   9.151227  7.419980          8.108021      7.850104
8     8.693329   8.201934   8.731013  6.052089          7.447751      6.620073
9     8.700514   9.314070   9.845911  7.055313          8.912608      7.648740
10    8.121480   8.594710   9.470703  8.389360          8.695674      7.463937
11    9.483873   7.024649   8.416931  7.258412          6.308098      6.208590
12   10.364514   9.418898   9.372204  5.659482          8.263848      7.983099
13    9.962558   8.733594   9.614605  8.037543          8.810907      6.400257
14   10.112654   9.155356   9.400217  5.683580          8.528726      7.681560
15    9.235326   7.015712   8.248267  5.983936          6.871091      6.021023
16    6.927558   9.084324   9.402695  4.897840          8.413609      6.984716
17    8.678632   8.725345   7.983781  6.732211          5.913503      8.406932
18    9.830971   8.752581   9.220192  7.698483          7.925519      8.064951
19    8.959312   7.822044   9.155250  6.505784          7.831220      6.216606
20    9.772581   8.416046   8.434246  6.971669          7.722678      7.661056
21    8.624612   6.769642   7.605890  8.126518          5.926926      6.343880
22   10.350606   7.558517   8.404920  9.149316          7.775276      8.374246
23   10.180096  10.502956   9.999661  8.547528          8.374938      9.712509
24   10.027783   9.187686   9.531844  7.977625          8.407825      8.661813
25    9.690604   8.349957   8.935245  5.303305          8.294799      4.043051
26    9.200088   6.867974   7.958926  8.055475          5.488938      6.725034
27    9.566335   6.688355   8.021256  6.184149          4.605170      6.249975
28    8.321908   9.927399  10.164197  7.054450          9.059982      8.557567
29   10.671000   7.649693   7.866722  7.090077          7.009409      6.712956
..         {\ldots}        {\ldots}        {\ldots}       {\ldots}               {\ldots}           {\ldots}
410   8.799812   7.647786   8.425736  7.236339          7.528332      7.545390
411   7.661998   8.098339   8.095904  7.336286          5.459586      8.381373
412   4.574711   8.190077   9.425452  4.584967          7.996317      4.127134
413   8.513787   8.488588   8.799812  9.790655          6.815640      7.797702
414   8.694335   7.595890   8.136518  8.644530          7.034388      5.669881
415   8.967249   8.707152   9.053920  7.433075          8.171882      7.535830
416   8.386857   9.300181   9.297252  6.742881          8.814033      6.900731
417   8.530109   8.612322   9.310638  5.897154          8.156223      6.968850
418   6.492240   9.047115   9.832099  4.890349          8.815815      6.654153
419   9.089415   8.238273   7.706613  6.450470          7.365180      7.327123
420   8.402007   8.569026   9.490015  3.218876          8.827321      7.239215
421   9.744668   8.486115   9.110851  6.938284          8.135933      7.486613
422  10.181119   7.227662   8.336151  6.721426          6.854355      7.104965
423   9.773664   8.212297   8.446127  6.965080          7.497207      6.504288
424   9.739791   7.966933   9.411811  6.773080          8.074960      5.517453
425   9.327501   7.786552   7.860571  9.638740          4.682131      7.542213
426   9.482960   9.142811   9.569133  8.052296          8.532870      7.546446
427  10.342130   9.722385   8.599510  9.621257          6.084499      7.058758
428   8.021913   8.694502   8.499029  7.695303          6.745236      5.758902
429   9.060331   7.467371   8.183118  3.850148          4.430817      7.824446
430   8.038189   8.349957   9.710085  6.354370          5.484797      7.640123
431   9.051696   8.613594   8.548692  9.509407          7.227662      7.311886
432   9.957834   7.057898   8.466742  5.594711          7.191429      5.978886
433   7.591862   8.076515   7.308543  7.340187          5.874931      7.278629
434   9.725019   8.274357   8.986447  6.533789          7.771067      6.731018
435  10.299003   9.396903   9.682030  9.483036          5.204007      7.698029
436  10.577146   7.266129   6.638568  8.414052          4.532599      7.760467
437   9.584040   9.647821  10.317020  6.079933          9.605149      7.532088
438   9.238928   7.591357   7.710653  6.945051          5.123964      7.661527
439   7.932721   7.437206   7.828038  4.174387          6.167516      3.951244

[430 rows x 6 columns]
Data points considered outliers for the feature 'Delicatessen':
         Fresh       Milk    Grocery    Frozen  Detergents\_Paper  Delicatessen
0     9.446913   9.175335   8.930759  5.365976          7.891331      7.198931
1     8.861775   9.191158   9.166179  7.474205          8.099554      7.482119
2     8.756682   9.083416   8.946896  7.785305          8.165079      8.967504
3     9.492884   7.086738   8.347827  8.764678          6.228511      7.488853
4    10.026369   8.596004   8.881558  8.272571          7.482682      8.553525
5     9.149847   9.019059   8.542081  6.501290          7.492760      7.280008
6     9.403107   8.070594   8.850088  6.173786          8.051978      6.300786
7     8.933137   8.508354   9.151227  7.419980          8.108021      7.850104
8     8.693329   8.201934   8.731013  6.052089          7.447751      6.620073
9     8.700514   9.314070   9.845911  7.055313          8.912608      7.648740
10    8.121480   8.594710   9.470703  8.389360          8.695674      7.463937
11    9.483873   7.024649   8.416931  7.258412          6.308098      6.208590
12   10.364514   9.418898   9.372204  5.659482          8.263848      7.983099
13    9.962558   8.733594   9.614605  8.037543          8.810907      6.400257
14   10.112654   9.155356   9.400217  5.683580          8.528726      7.681560
15    9.235326   7.015712   8.248267  5.983936          6.871091      6.021023
16    6.927558   9.084324   9.402695  4.897840          8.413609      6.984716
17    8.678632   8.725345   7.983781  6.732211          5.913503      8.406932
18    9.830971   8.752581   9.220192  7.698483          7.925519      8.064951
19    8.959312   7.822044   9.155250  6.505784          7.831220      6.216606
20    9.772581   8.416046   8.434246  6.971669          7.722678      7.661056
21    8.624612   6.769642   7.605890  8.126518          5.926926      6.343880
22   10.350606   7.558517   8.404920  9.149316          7.775276      8.374246
23   10.180096  10.502956   9.999661  8.547528          8.374938      9.712509
24   10.027783   9.187686   9.531844  7.977625          8.407825      8.661813
25    9.690604   8.349957   8.935245  5.303305          8.294799      4.043051
26    9.200088   6.867974   7.958926  8.055475          5.488938      6.725034
27    9.566335   6.688355   8.021256  6.184149          4.605170      6.249975
28    8.321908   9.927399  10.164197  7.054450          9.059982      8.557567
29   10.671000   7.649693   7.866722  7.090077          7.009409      6.712956
..         {\ldots}        {\ldots}        {\ldots}       {\ldots}               {\ldots}           {\ldots}
410   8.799812   7.647786   8.425736  7.236339          7.528332      7.545390
411   7.661998   8.098339   8.095904  7.336286          5.459586      8.381373
412   4.574711   8.190077   9.425452  4.584967          7.996317      4.127134
413   8.513787   8.488588   8.799812  9.790655          6.815640      7.797702
414   8.694335   7.595890   8.136518  8.644530          7.034388      5.669881
415   8.967249   8.707152   9.053920  7.433075          8.171882      7.535830
416   8.386857   9.300181   9.297252  6.742881          8.814033      6.900731
417   8.530109   8.612322   9.310638  5.897154          8.156223      6.968850
418   6.492240   9.047115   9.832099  4.890349          8.815815      6.654153
419   9.089415   8.238273   7.706613  6.450470          7.365180      7.327123
420   8.402007   8.569026   9.490015  3.218876          8.827321      7.239215
421   9.744668   8.486115   9.110851  6.938284          8.135933      7.486613
422  10.181119   7.227662   8.336151  6.721426          6.854355      7.104965
423   9.773664   8.212297   8.446127  6.965080          7.497207      6.504288
424   9.739791   7.966933   9.411811  6.773080          8.074960      5.517453
425   9.327501   7.786552   7.860571  9.638740          4.682131      7.542213
426   9.482960   9.142811   9.569133  8.052296          8.532870      7.546446
427  10.342130   9.722385   8.599510  9.621257          6.084499      7.058758
428   8.021913   8.694502   8.499029  7.695303          6.745236      5.758902
429   9.060331   7.467371   8.183118  3.850148          4.430817      7.824446
430   8.038189   8.349957   9.710085  6.354370          5.484797      7.640123
431   9.051696   8.613594   8.548692  9.509407          7.227662      7.311886
432   9.957834   7.057898   8.466742  5.594711          7.191429      5.978886
433   7.591862   8.076515   7.308543  7.340187          5.874931      7.278629
434   9.725019   8.274357   8.986447  6.533789          7.771067      6.731018
435  10.299003   9.396903   9.682030  9.483036          5.204007      7.698029
436  10.577146   7.266129   6.638568  8.414052          4.532599      7.760467
437   9.584040   9.647821  10.317020  6.079933          9.605149      7.532088
438   9.238928   7.591357   7.710653  6.945051          5.123964      7.661527
439   7.932721   7.437206   7.828038  4.174387          6.167516      3.951244

[436 rows x 6 columns]

    \end{Verbatim}

    \subsubsection{Question 4}\label{question-4}

\begin{itemize}
\tightlist
\item
  Are there any data points considered outliers for more than one
  feature based on the definition above?
\item
  Should these data points be removed from the dataset?
\item
  If any data points were added to the \texttt{outliers} list to be
  removed, explain why.
\end{itemize}

** Hint: ** If you have datapoints that are outliers in multiple
categories think about why that may be and if they warrant removal. Also
note how k-means is affected by outliers and whether or not this plays a
factor in your analysis of whether or not to remove them.

    \textbf{Answer:} data point with index 65 is an outlier for two features
Fresh,Frozen data point with index 66 is an outlier for two features
Fresh,Delicatessen data point with index 75 is an outlier for two
features Grocery and Detergent\_Paper data point with index 128 is an
outlier for two features Fresh,Delicatessen data point with index 154 is
an outlier for three features Grocery,Milk,Delicatessen We need to
remove these outliers because they appeared as outliers in more than one
case.If we use these outliers in train data it leads to false
clustering.

    \subsection{Feature Transformation}\label{feature-transformation}

In this section you will use principal component analysis (PCA) to draw
conclusions about the underlying structure of the wholesale customer
data. Since using PCA on a dataset calculates the dimensions which best
maximize variance, we will find which compound combinations of features
best describe customers.

    \subsubsection{Implementation: PCA}\label{implementation-pca}

Now that the data has been scaled to a more normal distribution and has
had any necessary outliers removed, we can now apply PCA to the
\texttt{good\_data} to discover which dimensions about the data best
maximize the variance of features involved. In addition to finding these
dimensions, PCA will also report the \emph{explained variance ratio} of
each dimension --- how much variance within the data is explained by
that dimension alone. Note that a component (dimension) from PCA can be
considered a new "feature" of the space, however it is a composition of
the original features present in the data.

In the code block below, you will need to implement the following: -
Import \texttt{sklearn.decomposition.PCA} and assign the results of
fitting PCA in six dimensions with \texttt{good\_data} to \texttt{pca}.
- Apply a PCA transformation of \texttt{log\_samples} using
\texttt{pca.transform}, and assign the results to \texttt{pca\_samples}.

    \begin{Verbatim}[commandchars=\\\{\}]
{\color{incolor}In [{\color{incolor}204}]:} \PY{c+c1}{\PYZsh{} TODO: Apply PCA by fitting the good data with the same number of dimensions as features}
          \PY{k+kn}{from} \PY{n+nn}{sklearn}\PY{n+nn}{.}\PY{n+nn}{decomposition} \PY{k}{import} \PY{n}{PCA}
          \PY{n}{pca} \PY{o}{=} \PY{n}{PCA}\PY{p}{(}\PY{n}{n\PYZus{}components} \PY{o}{=} \PY{l+m+mi}{6}\PY{p}{)}
          \PY{n}{pca}\PY{o}{.}\PY{n}{fit}\PY{p}{(}\PY{n}{good\PYZus{}data}\PY{p}{)}
          
          \PY{c+c1}{\PYZsh{} TODO: Transform log\PYZus{}samples using the PCA fit above}
          \PY{n}{pca\PYZus{}samples} \PY{o}{=} \PY{n}{pca}\PY{o}{.}\PY{n}{transform}\PY{p}{(}\PY{n}{log\PYZus{}samples}\PY{p}{)}
          
          \PY{c+c1}{\PYZsh{} Generate PCA results plot}
          \PY{n}{pca\PYZus{}results} \PY{o}{=} \PY{n}{vs}\PY{o}{.}\PY{n}{pca\PYZus{}results}\PY{p}{(}\PY{n}{good\PYZus{}data}\PY{p}{,} \PY{n}{pca}\PY{p}{)}
\end{Verbatim}


    \begin{center}
    \adjustimage{max size={0.9\linewidth}{0.9\paperheight}}{output_31_0.png}
    \end{center}
    { \hspace*{\fill} \\}
    
    \subsubsection{Question 5}\label{question-5}

\begin{itemize}
\tightlist
\item
  How much variance in the data is explained* \textbf{in total} *by the
  first and second principal component?
\item
  How much variance in the data is explained by the first four principal
  components?
\item
  Using the visualization provided above, talk about each dimension and
  the cumulative variance explained by each, stressing upon which
  features are well represented by each dimension(both in terms of
  positive and negative variance explained). Discuss what the first four
  dimensions best represent in terms of customer spending.
\end{itemize}

\textbf{Hint:} A positive increase in a specific dimension corresponds
with an \emph{increase} of the \emph{positive-weighted} features and a
\emph{decrease} of the \emph{negative-weighted} features. The rate of
increase or decrease is based on the individual feature weights.

    \textbf{Answer:} Variance in the data explained by first and second
principal component is 0.4424+0.2766 = 0.719 For Four principal
components explained variance is 0.719+0.1162+0.0962 = 0.9314 First
principal component has a large negative emphasis is placed on
Detergents\_Paper with a smaller yet large negative emphasis on the
features Grocery and Milk, this kind of spending is best represented by
spending on retail goods. Second principal component has a large
negattive weights on Fresh, Frozen and Delicatessen. This pattern might
represent spending in frozen fresh fruit and deli products Third
principal component has large positive weights on Delicatessen. This
pattern might represent spending in deli products Fourth principal
component has large positive weights on Frozen It also correlates with a
decrease in Delicatessen. This pattern might represent spending in
frozen products.

    \subsubsection{Observation}\label{observation}

Run the code below to see how the log-transformed sample data has
changed after having a PCA transformation applied to it in six
dimensions. Observe the numerical value for the first four dimensions of
the sample points. Consider if this is consistent with your initial
interpretation of the sample points.

    \begin{Verbatim}[commandchars=\\\{\}]
{\color{incolor}In [{\color{incolor}205}]:} \PY{c+c1}{\PYZsh{} Display sample log\PYZhy{}data after having a PCA transformation applied}
          \PY{n}{display}\PY{p}{(}\PY{n}{pd}\PY{o}{.}\PY{n}{DataFrame}\PY{p}{(}\PY{n}{np}\PY{o}{.}\PY{n}{round}\PY{p}{(}\PY{n}{pca\PYZus{}samples}\PY{p}{,} \PY{l+m+mi}{4}\PY{p}{)}\PY{p}{,} \PY{n}{columns} \PY{o}{=} \PY{n}{pca\PYZus{}results}\PY{o}{.}\PY{n}{index}\PY{o}{.}\PY{n}{values}\PY{p}{)}\PY{p}{)}
\end{Verbatim}


    
    \begin{verbatim}
   Dimension 1  Dimension 2  Dimension 3  Dimension 4  Dimension 5  \
0      -2.3579      -1.7393       0.2210       0.2840      -0.5939   
1       1.9406      -0.2418      -0.2884      -1.2041       0.0917   
2      -1.2804      -0.9587      -0.4701      -0.9124      -0.2345   

   Dimension 6  
0      -0.0148  
1      -0.1492  
2      -0.2514  
    \end{verbatim}

    
    \subsubsection{Implementation: Dimensionality
Reduction}\label{implementation-dimensionality-reduction}

When using principal component analysis, one of the main goals is to
reduce the dimensionality of the data --- in effect, reducing the
complexity of the problem. Dimensionality reduction comes at a cost:
Fewer dimensions used implies less of the total variance in the data is
being explained. Because of this, the \emph{cumulative explained
variance ratio} is extremely important for knowing how many dimensions
are necessary for the problem. Additionally, if a signifiant amount of
variance is explained by only two or three dimensions, the reduced data
can be visualized afterwards.

In the code block below, you will need to implement the following: -
Assign the results of fitting PCA in two dimensions with
\texttt{good\_data} to \texttt{pca}. - Apply a PCA transformation of
\texttt{good\_data} using \texttt{pca.transform}, and assign the results
to \texttt{reduced\_data}. - Apply a PCA transformation of
\texttt{log\_samples} using \texttt{pca.transform}, and assign the
results to \texttt{pca\_samples}.

    \begin{Verbatim}[commandchars=\\\{\}]
{\color{incolor}In [{\color{incolor}206}]:} \PY{c+c1}{\PYZsh{} TODO: Apply PCA by fitting the good data with only two dimensions}
          \PY{n}{pca} \PY{o}{=} \PY{n}{PCA}\PY{p}{(}\PY{n}{n\PYZus{}components} \PY{o}{=} \PY{l+m+mi}{2} \PY{p}{)}
          \PY{n}{pca}\PY{o}{.}\PY{n}{fit}\PY{p}{(}\PY{n}{good\PYZus{}data}\PY{p}{)}
          \PY{c+c1}{\PYZsh{} TODO: Transform the good data using the PCA fit above}
          \PY{n}{reduced\PYZus{}data} \PY{o}{=} \PY{n}{pca}\PY{o}{.}\PY{n}{transform}\PY{p}{(}\PY{n}{good\PYZus{}data}\PY{p}{)}
          
          \PY{c+c1}{\PYZsh{} TODO: Transform log\PYZus{}samples using the PCA fit above}
          \PY{n}{pca\PYZus{}samples} \PY{o}{=} \PY{n}{pca}\PY{o}{.}\PY{n}{transform}\PY{p}{(}\PY{n}{log\PYZus{}samples}\PY{p}{)}
          
          \PY{c+c1}{\PYZsh{} Create a DataFrame for the reduced data}
          \PY{n}{reduced\PYZus{}data} \PY{o}{=} \PY{n}{pd}\PY{o}{.}\PY{n}{DataFrame}\PY{p}{(}\PY{n}{reduced\PYZus{}data}\PY{p}{,} \PY{n}{columns} \PY{o}{=} \PY{p}{[}\PY{l+s+s1}{\PYZsq{}}\PY{l+s+s1}{Dimension 1}\PY{l+s+s1}{\PYZsq{}}\PY{p}{,} \PY{l+s+s1}{\PYZsq{}}\PY{l+s+s1}{Dimension 2}\PY{l+s+s1}{\PYZsq{}}\PY{p}{]}\PY{p}{)}
\end{Verbatim}


    \subsubsection{Observation}\label{observation}

Run the code below to see how the log-transformed sample data has
changed after having a PCA transformation applied to it using only two
dimensions. Observe how the values for the first two dimensions remains
unchanged when compared to a PCA transformation in six dimensions.

    \begin{Verbatim}[commandchars=\\\{\}]
{\color{incolor}In [{\color{incolor}207}]:} \PY{c+c1}{\PYZsh{} Display sample log\PYZhy{}data after applying PCA transformation in two dimensions}
          \PY{n}{display}\PY{p}{(}\PY{n}{pd}\PY{o}{.}\PY{n}{DataFrame}\PY{p}{(}\PY{n}{np}\PY{o}{.}\PY{n}{round}\PY{p}{(}\PY{n}{pca\PYZus{}samples}\PY{p}{,} \PY{l+m+mi}{4}\PY{p}{)}\PY{p}{,} \PY{n}{columns} \PY{o}{=} \PY{p}{[}\PY{l+s+s1}{\PYZsq{}}\PY{l+s+s1}{Dimension 1}\PY{l+s+s1}{\PYZsq{}}\PY{p}{,} \PY{l+s+s1}{\PYZsq{}}\PY{l+s+s1}{Dimension 2}\PY{l+s+s1}{\PYZsq{}}\PY{p}{]}\PY{p}{)}\PY{p}{)}
\end{Verbatim}


    
    \begin{verbatim}
   Dimension 1  Dimension 2
0      -2.3579      -1.7393
1       1.9406      -0.2418
2      -1.2804      -0.9587
    \end{verbatim}

    
    \subsection{Visualizing a Biplot}\label{visualizing-a-biplot}

A biplot is a scatterplot where each data point is represented by its
scores along the principal components. The axes are the principal
components (in this case \texttt{Dimension\ 1} and
\texttt{Dimension\ 2}). In addition, the biplot shows the projection of
the original features along the components. A biplot can help us
interpret the reduced dimensions of the data, and discover relationships
between the principal components and original features.

Run the code cell below to produce a biplot of the reduced-dimension
data.

    \begin{Verbatim}[commandchars=\\\{\}]
{\color{incolor}In [{\color{incolor}208}]:} \PY{c+c1}{\PYZsh{} Create a biplot}
          \PY{n}{vs}\PY{o}{.}\PY{n}{biplot}\PY{p}{(}\PY{n}{good\PYZus{}data}\PY{p}{,} \PY{n}{reduced\PYZus{}data}\PY{p}{,} \PY{n}{pca}\PY{p}{)}
\end{Verbatim}


\begin{Verbatim}[commandchars=\\\{\}]
{\color{outcolor}Out[{\color{outcolor}208}]:} <matplotlib.axes.\_subplots.AxesSubplot at 0x29a3d2680b8>
\end{Verbatim}
            
    \begin{center}
    \adjustimage{max size={0.9\linewidth}{0.9\paperheight}}{output_41_1.png}
    \end{center}
    { \hspace*{\fill} \\}
    
    \subsubsection{Observation}\label{observation}

Once we have the original feature projections (in red), it is easier to
interpret the relative position of each data point in the scatterplot.
For instance, a point the lower right corner of the figure will likely
correspond to a customer that spends a lot on
\texttt{\textquotesingle{}Milk\textquotesingle{}},
\texttt{\textquotesingle{}Grocery\textquotesingle{}} and
\texttt{\textquotesingle{}Detergents\_Paper\textquotesingle{}}, but not
so much on the other product categories.

From the biplot, which of the original features are most strongly
correlated with the first component? What about those that are
associated with the second component? Do these observations agree with
the pca\_results plot you obtained earlier?

    \subsection{Clustering}\label{clustering}

In this section, you will choose to use either a K-Means clustering
algorithm or a Gaussian Mixture Model clustering algorithm to identify
the various customer segments hidden in the data. You will then recover
specific data points from the clusters to understand their significance
by transforming them back into their original dimension and scale.

    \subsubsection{Question 6}\label{question-6}

\begin{itemize}
\tightlist
\item
  What are the advantages to using a K-Means clustering algorithm?
\item
  What are the advantages to using a Gaussian Mixture Model clustering
  algorithm?
\item
  Given your observations about the wholesale customer data so far,
  which of the two algorithms will you use and why?
\end{itemize}

** Hint: ** Think about the differences between hard clustering and soft
clustering and which would be appropriate for our dataset.

    \textbf{Answer:} K-Means clustering algorithm: Points will belong to a
cluster soft assignment.It can be done By minimize the distance within
same cluster.KMeans algorithm is fast. but it falls in local minima.
That's why it can be useful to restart it several times.It is
susceptible to outliers. We can pre-process our data to exclude outliers
to solve this. Gaussian Mixture Model clustering: Points have
probabilities of belonging to different clusters.It is hard
clustering.It is complicated model to initialize. Here I am using KMeans
clustering we can decide number of clusters by using silhouette
coefficient.We alredy removed outliers also it will perform well for the
given data

    \subsubsection{Implementation: Creating
Clusters}\label{implementation-creating-clusters}

Depending on the problem, the number of clusters that you expect to be
in the data may already be known. When the number of clusters is not
known \emph{a priori}, there is no guarantee that a given number of
clusters best segments the data, since it is unclear what structure
exists in the data --- if any. However, we can quantify the "goodness"
of a clustering by calculating each data point's \emph{silhouette
coefficient}. The
\href{http://scikit-learn.org/stable/modules/generated/sklearn.metrics.silhouette_score.html}{silhouette
coefficient} for a data point measures how similar it is to its assigned
cluster from -1 (dissimilar) to 1 (similar). Calculating the \emph{mean}
silhouette coefficient provides for a simple scoring method of a given
clustering.

In the code block below, you will need to implement the following: - Fit
a clustering algorithm to the \texttt{reduced\_data} and assign it to
\texttt{clusterer}. - Predict the cluster for each data point in
\texttt{reduced\_data} using \texttt{clusterer.predict} and assign them
to \texttt{preds}. - Find the cluster centers using the algorithm's
respective attribute and assign them to \texttt{centers}. - Predict the
cluster for each sample data point in \texttt{pca\_samples} and assign
them \texttt{sample\_preds}. - Import
\texttt{sklearn.metrics.silhouette\_score} and calculate the silhouette
score of \texttt{reduced\_data} against \texttt{preds}. - Assign the
silhouette score to \texttt{score} and print the result.

    \begin{Verbatim}[commandchars=\\\{\}]
{\color{incolor}In [{\color{incolor}209}]:} \PY{c+c1}{\PYZsh{} TODO: Apply your clustering algorithm of choice to the reduced data }
          \PY{k+kn}{from} \PY{n+nn}{sklearn}\PY{n+nn}{.}\PY{n+nn}{cluster} \PY{k}{import} \PY{n}{KMeans}
          \PY{k+kn}{from} \PY{n+nn}{sklearn}\PY{n+nn}{.}\PY{n+nn}{metrics} \PY{k}{import} \PY{n}{silhouette\PYZus{}score}
          \PY{n}{clusterer} \PY{o}{=} \PY{n}{KMeans}\PY{p}{(}\PY{n}{n\PYZus{}clusters}\PY{o}{=}\PY{l+m+mi}{2}\PY{p}{,}\PY{n}{random\PYZus{}state} \PY{o}{=}\PY{l+m+mi}{42}\PY{p}{)}\PY{o}{.}\PY{n}{fit}\PY{p}{(}\PY{n}{reduced\PYZus{}data}\PY{p}{)}
          
          \PY{c+c1}{\PYZsh{} TODO: Predict the cluster for each data point}
          \PY{n}{preds} \PY{o}{=} \PY{n}{clusterer}\PY{o}{.}\PY{n}{predict}\PY{p}{(}\PY{n}{reduced\PYZus{}data}\PY{p}{)}
          
          \PY{c+c1}{\PYZsh{} TODO: Find the cluster centers}
          \PY{n}{centers} \PY{o}{=} \PY{n}{clusterer}\PY{o}{.}\PY{n}{cluster\PYZus{}centers\PYZus{}}
          
          \PY{c+c1}{\PYZsh{} TODO: Predict the cluster for each transformed sample data point}
          \PY{n}{sample\PYZus{}preds} \PY{o}{=} \PY{n}{clusterer}\PY{o}{.}\PY{n}{predict}\PY{p}{(}\PY{n}{pca\PYZus{}samples}\PY{p}{)}
          
          \PY{c+c1}{\PYZsh{} TODO: Calculate the mean silhouette coefficient for the number of clusters chosen}
          \PY{n}{score} \PY{o}{=} \PY{n}{silhouette\PYZus{}score}\PY{p}{(}\PY{n}{reduced\PYZus{}data}\PY{p}{,}\PY{n}{preds}\PY{p}{)} 
          \PY{n}{score}
\end{Verbatim}


\begin{Verbatim}[commandchars=\\\{\}]
{\color{outcolor}Out[{\color{outcolor}209}]:} 0.4262810154691084
\end{Verbatim}
            
    \subsubsection{Question 7}\label{question-7}

\begin{itemize}
\tightlist
\item
  Report the silhouette score for several cluster numbers you tried.
\item
  Of these, which number of clusters has the best silhouette score?
\end{itemize}

    \textbf{Answer:} For cluster numbers 2 silhouette score is
0.4262810154691084 For cluster numbers 3 silhouette score is
0.39689092644980506 For cluster numbers 4 silhouette score is
0.3318412760093694 For cluster numbers 5 silhouette score is
0.34999779752629756 Here for number of clusters 2 we got highest
silhouette score so this the best.

    \subsubsection{Cluster Visualization}\label{cluster-visualization}

Once you've chosen the optimal number of clusters for your clustering
algorithm using the scoring metric above, you can now visualize the
results by executing the code block below. Note that, for
experimentation purposes, you are welcome to adjust the number of
clusters for your clustering algorithm to see various visualizations.
The final visualization provided should, however, correspond with the
optimal number of clusters.

    \begin{Verbatim}[commandchars=\\\{\}]
{\color{incolor}In [{\color{incolor}210}]:} \PY{c+c1}{\PYZsh{} Display the results of the clustering from implementation}
          \PY{n}{vs}\PY{o}{.}\PY{n}{cluster\PYZus{}results}\PY{p}{(}\PY{n}{reduced\PYZus{}data}\PY{p}{,} \PY{n}{preds}\PY{p}{,} \PY{n}{centers}\PY{p}{,} \PY{n}{pca\PYZus{}samples}\PY{p}{)}
\end{Verbatim}


    \begin{center}
    \adjustimage{max size={0.9\linewidth}{0.9\paperheight}}{output_51_0.png}
    \end{center}
    { \hspace*{\fill} \\}
    
    \subsubsection{Implementation: Data
Recovery}\label{implementation-data-recovery}

Each cluster present in the visualization above has a central point.
These centers (or means) are not specifically data points from the data,
but rather the \emph{averages} of all the data points predicted in the
respective clusters. For the problem of creating customer segments, a
cluster's center point corresponds to \emph{the average customer of that
segment}. Since the data is currently reduced in dimension and scaled by
a logarithm, we can recover the representative customer spending from
these data points by applying the inverse transformations.

In the code block below, you will need to implement the following: -
Apply the inverse transform to \texttt{centers} using
\texttt{pca.inverse\_transform} and assign the new centers to
\texttt{log\_centers}. - Apply the inverse function of \texttt{np.log}
to \texttt{log\_centers} using \texttt{np.exp} and assign the true
centers to \texttt{true\_centers}.

    \begin{Verbatim}[commandchars=\\\{\}]
{\color{incolor}In [{\color{incolor}211}]:} \PY{c+c1}{\PYZsh{} TODO: Inverse transform the centers}
          \PY{n}{log\PYZus{}centers} \PY{o}{=} \PY{n}{pca}\PY{o}{.}\PY{n}{inverse\PYZus{}transform}\PY{p}{(}\PY{n}{centers}\PY{p}{)}
          
          \PY{c+c1}{\PYZsh{} TODO: Exponentiate the centers}
          \PY{n}{true\PYZus{}centers} \PY{o}{=} \PY{n}{np}\PY{o}{.}\PY{n}{exp}\PY{p}{(}\PY{n}{log\PYZus{}centers}\PY{p}{)}
          
          \PY{c+c1}{\PYZsh{} Display the true centers}
          \PY{n}{segments} \PY{o}{=} \PY{p}{[}\PY{l+s+s1}{\PYZsq{}}\PY{l+s+s1}{Segment }\PY{l+s+si}{\PYZob{}\PYZcb{}}\PY{l+s+s1}{\PYZsq{}}\PY{o}{.}\PY{n}{format}\PY{p}{(}\PY{n}{i}\PY{p}{)} \PY{k}{for} \PY{n}{i} \PY{o+ow}{in} \PY{n+nb}{range}\PY{p}{(}\PY{l+m+mi}{0}\PY{p}{,}\PY{n+nb}{len}\PY{p}{(}\PY{n}{centers}\PY{p}{)}\PY{p}{)}\PY{p}{]}
          \PY{n}{true\PYZus{}centers} \PY{o}{=} \PY{n}{pd}\PY{o}{.}\PY{n}{DataFrame}\PY{p}{(}\PY{n}{np}\PY{o}{.}\PY{n}{round}\PY{p}{(}\PY{n}{true\PYZus{}centers}\PY{p}{)}\PY{p}{,} \PY{n}{columns} \PY{o}{=} \PY{n}{data}\PY{o}{.}\PY{n}{keys}\PY{p}{(}\PY{p}{)}\PY{p}{)}
          \PY{n}{true\PYZus{}centers}\PY{o}{.}\PY{n}{index} \PY{o}{=} \PY{n}{segments}
          \PY{n}{display}\PY{p}{(}\PY{n}{true\PYZus{}centers}\PY{p}{)}
\end{Verbatim}


    
    \begin{verbatim}
            Fresh    Milk  Grocery  Frozen  Detergents_Paper  Delicatessen
Segment 0  8867.0  1897.0   2477.0  2088.0             294.0         681.0
Segment 1  4005.0  7900.0  12104.0   952.0            4561.0        1036.0
    \end{verbatim}

    
    \subsubsection{Question 8}\label{question-8}

\begin{itemize}
\tightlist
\item
  Consider the total purchase cost of each product category for the
  representative data points above, and reference the statistical
  description of the dataset at the beginning of this
  project(specifically looking at the mean values for the various
  feature points). What set of establishments could each of the customer
  segments represent?
\end{itemize}

\textbf{Hint:} A customer who is assigned to
\texttt{\textquotesingle{}Cluster\ X\textquotesingle{}} should best
identify with the establishments represented by the feature set of
\texttt{\textquotesingle{}Segment\ X\textquotesingle{}}. Think about
what each segment represents in terms their values for the feature
points chosen. Reference these values with the mean values to get some
perspective into what kind of establishment they represent.

    \textbf{Answer:} In case of Segment 0 fresh and Frozen features total
cost values are near mean.So,Ice cream ice cream parlour,flower
market,vegetable market all of them will comes under this segment. In
case of segment 1 Milk ,Grocery and Detergent paper has total cost above
average.So,Super markets,Grocery stores comes under this category.

    \subsubsection{Question 9}\label{question-9}

\begin{itemize}
\tightlist
\item
  For each sample point, which customer segment from* \textbf{Question
  8} *best represents it?
\item
  Are the predictions for each sample point consistent with this?*
\end{itemize}

Run the code block below to find which cluster each sample point is
predicted to be.

    \begin{Verbatim}[commandchars=\\\{\}]
{\color{incolor}In [{\color{incolor}212}]:} \PY{c+c1}{\PYZsh{} Display the predictions}
          \PY{k}{for} \PY{n}{i}\PY{p}{,} \PY{n}{pred} \PY{o+ow}{in} \PY{n+nb}{enumerate}\PY{p}{(}\PY{n}{sample\PYZus{}preds}\PY{p}{)}\PY{p}{:}
              \PY{n+nb}{print}\PY{p}{(}\PY{l+s+s2}{\PYZdq{}}\PY{l+s+s2}{Sample point}\PY{l+s+s2}{\PYZdq{}}\PY{p}{,} \PY{n}{i}\PY{p}{,} \PY{l+s+s2}{\PYZdq{}}\PY{l+s+s2}{predicted to be in Cluster}\PY{l+s+s2}{\PYZdq{}}\PY{p}{,} \PY{n}{pred}\PY{p}{)}
\end{Verbatim}


    \begin{Verbatim}[commandchars=\\\{\}]
Sample point 0 predicted to be in Cluster 1
Sample point 1 predicted to be in Cluster 0
Sample point 2 predicted to be in Cluster 1

    \end{Verbatim}

    \textbf{Answer:} For sample 0 segment 1 best fits it because all 6
products cost is above mean.which mostly fall into segment 1. For sample
1 fresh cost is more than its mean all other less than their mean value
which comes under segemnt 0. For sample 2 fresh cost is more than its
mean all other less than their mean value which comes under segemnt 1.
Predictions for this samples are consistent with the intution

    \subsection{Conclusion}\label{conclusion}

    In this final section, you will investigate ways that you can make use
of the clustered data. First, you will consider how the different groups
of customers, the \textbf{\emph{customer segments}}, may be affected
differently by a specific delivery scheme. Next, you will consider how
giving a label to each customer (which \emph{segment} that customer
belongs to) can provide for additional features about the customer data.
Finally, you will compare the \textbf{\emph{customer segments}} to a
hidden variable present in the data, to see whether the clustering
identified certain relationships.

    \subsubsection{Question 10}\label{question-10}

Companies will often run
\href{https://en.wikipedia.org/wiki/A/B_testing}{A/B tests} when making
small changes to their products or services to determine whether making
that change will affect its customers positively or negatively. The
wholesale distributor is considering changing its delivery service from
currently 5 days a week to 3 days a week. However, the distributor will
only make this change in delivery service for customers that react
positively.

\begin{itemize}
\tightlist
\item
  How can the wholesale distributor use the customer segments to
  determine which customers, if any, would react positively to the
  change in delivery service?*
\end{itemize}

\textbf{Hint:} Can we assume the change affects all customers equally?
How can we determine which group of customers it affects the most?

    \textbf{Answer:} We can test any one of the segment first by changing
delivery services to 3days week.If they respond positively then we can
continue this service for this particular segment otherwise cancel the
new service to this segment.Then we will run the same process for second
segment. By doing this we can reduce the testing data size we can
provide specific services to the needed reatailers only.

    \subsubsection{Question 11}\label{question-11}

Additional structure is derived from originally unlabeled data when
using clustering techniques. Since each customer has a
\textbf{\emph{customer segment}} it best identifies with (depending on
the clustering algorithm applied), we can consider \emph{'customer
segment'} as an \textbf{engineered feature} for the data. Assume the
wholesale distributor recently acquired ten new customers and each
provided estimates for anticipated annual spending of each product
category. Knowing these estimates, the wholesale distributor wants to
classify each new customer to a \textbf{\emph{customer segment}} to
determine the most appropriate delivery service.\\
* How can the wholesale distributor label the new customers using only
their estimated product spending and the \textbf{customer segment} data?

\textbf{Hint:} A supervised learner could be used to train on the
original customers. What would be the target variable?

    \textbf{Answer:} Here Now we divided our data into two clusters.Now we
alredy know the labeling for each data point as 0 or 1.I can test the
new data with this and clasify into any one of the category 0 or 1.

    \subsubsection{Visualizing Underlying
Distributions}\label{visualizing-underlying-distributions}

At the beginning of this project, it was discussed that the
\texttt{\textquotesingle{}Channel\textquotesingle{}} and
\texttt{\textquotesingle{}Region\textquotesingle{}} features would be
excluded from the dataset so that the customer product categories were
emphasized in the analysis. By reintroducing the
\texttt{\textquotesingle{}Channel\textquotesingle{}} feature to the
dataset, an interesting structure emerges when considering the same PCA
dimensionality reduction applied earlier to the original dataset.

Run the code block below to see how each data point is labeled either
\texttt{\textquotesingle{}HoReCa\textquotesingle{}}
(Hotel/Restaurant/Cafe) or
\texttt{\textquotesingle{}Retail\textquotesingle{}} the reduced space.
In addition, you will find the sample points are circled in the plot,
which will identify their labeling.

    \begin{Verbatim}[commandchars=\\\{\}]
{\color{incolor}In [{\color{incolor}213}]:} \PY{c+c1}{\PYZsh{} Display the clustering results based on \PYZsq{}Channel\PYZsq{} data}
          \PY{n}{vs}\PY{o}{.}\PY{n}{channel\PYZus{}results}\PY{p}{(}\PY{n}{reduced\PYZus{}data}\PY{p}{,} \PY{n}{outliers}\PY{p}{,} \PY{n}{pca\PYZus{}samples}\PY{p}{)}
\end{Verbatim}


    \begin{center}
    \adjustimage{max size={0.9\linewidth}{0.9\paperheight}}{output_66_0.png}
    \end{center}
    { \hspace*{\fill} \\}
    
    \subsubsection{Question 12}\label{question-12}

\begin{itemize}
\tightlist
\item
  How well does the clustering algorithm and number of clusters you've
  chosen compare to this underlying distribution of
  Hotel/Restaurant/Cafe customers to Retailer customers?
\item
  Are there customer segments that would be classified as purely
  'Retailers' or 'Hotels/Restaurants/Cafes' by this distribution?
\item
  Would you consider these classifications as consistent with your
  previous definition of the customer segments?
\end{itemize}

    \textbf{Answer:} For this particular case whether customers are
Hotel/Restaurant/Cafe customers or Retailer customers,Thaere are only
two categories.That means we only need two clusters. yes,There are
customer segments that are clasiffied as purely 'Retailers' or
'Hotels/Restaurants/Cafes'. Cluster 1 is for Hotels/Restaurants/Cafes.
Cluster 0 is for Retailers These classifications are consistent with my
previous definitinon of customer segment in the previous parts.

    \begin{quote}
\textbf{Note}: Once you have completed all of the code implementations
and successfully answered each question above, you may finalize your
work by exporting the iPython Notebook as an HTML document. You can do
this by using the menu above and navigating to\\
\textbf{File -\textgreater{} Download as -\textgreater{} HTML (.html)}.
Include the finished document along with this notebook as your
submission.
\end{quote}


    % Add a bibliography block to the postdoc
    
    
    
    \end{document}
